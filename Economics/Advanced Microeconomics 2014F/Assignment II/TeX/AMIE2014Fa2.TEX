\documentclass{article}
  \usepackage{amsmath}
  \usepackage{amssymb}
  \usepackage{graphicx}
  \usepackage{float}
  \usepackage{setspace}
  \usepackage{stata}
\topmargin=-1.2cm \oddsidemargin=0.1cm \evensidemargin=0.1cm
\textwidth=16 true cm \textheight=23 true cm

\font\euler=EUSM10 \font\eulers=EUSM7

\begin{document}
\title{Advanced Microeconomics\\Assignment $2^{\text{nd}}$}
\author{{\normalsize SHENG Hao, 1401211818, via \LaTeX}}
\date{\today}

\maketitle

\def \Pr{{\rm Pr}}
\baselineskip 0.6cm
\section{Q1}

This proposition can be extended to a $n$ dimension goods edition:
For a two budget constrains $P^T_1X = I$ and $P^T_2X = I$, where $P^T_i(i=1,2)$ is a $n$ dimension row vector of price and $X$ is a $n$ dimension column vector of goods, if they intercept at some point, say $X_0$, then $X_0$ is also on the hyperplane of liner combination of price system $P^T_1$ and $P^T_2$. Moreover, the new hyperplane is always bounded by the original two budget constraints.

Suppose we have:
\begin{equation}
P^T_1X_0 = I
\end{equation}
and
\begin{equation}
P^T_2X_0 = I
\end{equation}
Given that $P^T = \lambda P^T_1 + (1-\lambda) P^T_2$, we have:
\begin{equation}
P^TX_0 = \lambda P^T_1 X_0 + (1-\lambda) P^T_2 = I
\end{equation}

\section{Q2\footnote{MWG.3.G.15$^\text{B}$}}

\section{Q3}

\section{Q4}

\section{Q5}


\newpage
\begin{appendix}
\section*{Appendix}
\subsection{Stata Do-file}
\subsection{Stata Log-file}
\end{appendix}
\end{document}

\documentclass{article}
  \usepackage{amsmath}
  \usepackage{amssymb}
  \usepackage{graphicx}
  \usepackage{float}
  \usepackage{setspace}
  \usepackage{stata}
  \usepackage{bm}
  \usepackage{amssymb}
  \usepackage{listings}
\topmargin=-1.2cm \oddsidemargin=0.1cm \evensidemargin=0.1cm
\textwidth=16 true cm \textheight=23 true cm

\font\euler=EUSM10 \font\eulers=EUSM7

\begin{document}
\title{Advanced Microeconomics\\Assignment $4^{\text{th}}$}
\author{{\normalsize SHENG Hao, 1401211818, via \LaTeX}}
\date{\today}

\maketitle

\def \Pr{{\rm Pr}}
\baselineskip 0.6cm
\section*{Q1}
Suppose the individual's total amount of asset is $A$, and his allocation between two assets is $\left(A_1,A_2\right)$, where $A_1+A_2=A$. The individual maximizes his expected utility level, which is
\begin{equation}
	E\left[u\left(A_1 R_1+A_2 R_2+A\right)\right]
\end{equation}
First let's suppose the individual is risk averse, i.e.
\begin{equation}
	u^{\prime\prime}(w) < 0
\end{equation}
Since $R_1$ and $R_2$ satisfy the same distribution (with CDF $f(.)$), the expected utility can be written as:
\begin{equation}
	\label{eq_1_1_int_trans}
	\begin{aligned}
		E\left[u\left(A_1 R_1+A_2 R_2+A\right)\right] & = E\left[u\left(A_2 R_1+A_1 R_2+A\right)\right] \\
		& = \frac{1}{2} \left( E\left[u\left(A_1 R_1+A_2 R_2+A\right)\right] +  E\left[u\left(A_2 R_1+A_1 R_2+A\right)\right] \right) \\
		& = E \left( \frac{1}{2} \left[ u(A+A_1 R_1 + A_2 R_2) + u(A+A_2 R_1 + A_1 R_2) \right] \right)
	\end{aligned}
\end{equation}
Note that the individual is risk averse, hence $u^{\prime\prime}(w) < 0$, we must have
\begin{equation}
	\label{eq_1_1_neq}
	\frac{1}{2} \left[ u(A+A_1 R_1 + A_2 R_2) + u(A+A_2 R_1 + A_1 R_2) \right] \leq u \left( A+\frac{1}{2}A R_1 + \frac{1}{2}A R_2 \right)
\end{equation}
This simply follows from the property of concave functions and the fact that $A_1 + A_2 = A$. Note \eqref{eq_1_1_neq} holds for any allocation $(A_1, A_2)$ and return $(R_1, R_2)$. Thus, from \eqref{eq_1_1_int_trans} and \eqref{eq_1_1_neq}, we must have:
\begin{equation}
	E\left[u\left(A_1 R_1+A_2 R_2+A\right)\right] \leq E\left[u\left(\frac{1}{2} A R_1+  \frac{1}{2} A R_2+A\right)\right]
\end{equation}
 Thus we can conclude that if the individual is risk averse, then he maximize his expected utility when he evenly allocates his money between two assets.
\par
Now let's suppose the individual is risk loving. It then follows that
\begin{equation}
	u^{\prime\prime}(w) > 0
\end{equation}
From the property of convex functions, we have:
\begin{equation}
	\frac{A_1}{A}u(A+A R_1) + \frac{A_2}{A}u(A+A R_2) \geq u(A+A_1 R_1 + A_2 R_2)
\end{equation}
This also holds for any allocation $(A_1, A_2)$ and return $(R_1, R_2)$. Hence, we have:
\begin{equation}
	\label{eq_1_2_neq}
	\frac{A_1}{A}E\left[ u(A+A R_1) \right] + \frac{A_2}{A} E \left[ u(A+A R_2) \right] \geq E \left[ u(A+A_1 R_1 + A_2 R_2) \right]
\end{equation}
Since $R_1$ and $R_2$ satisfy the same distribution, we have:
\begin{equation}
	E\left[ u(A+A R_1) \right] = E \left[ u(A+A R_2) \right]
\end{equation}
Thus \eqref{eq_1_2_neq} can be written as
\begin{equation}
	E\left[ u(A+A R_1) \right] \geq E \left[ u(A+A_1 R_1 + A_2 R_2) \right]
\end{equation}
This holds for any allocation $(A_1, A_2)$. Thus, when the individual is risk loving, the expected utility is maximized when the individual allocates all his money on one asset.$\blacksquare$

\section*{Q2}
\subsection*{1)}
If the individual owns the lottery, suppose the price he would sell it for is $S$. Then the expected utility if he sells the lottery would be:
	\begin{equation}
		u = u (w+S)
	\end{equation}
	On the other hand, if he does not sell the lottery, the expected utility would be:
	\begin{equation}
		u = p \cdot u(w+G) + (1-p) \cdot u(w+B)
	\end{equation}
	The minimum price he would sell the lottery for, $S^{*}$, is the price which makes the individual indifferent about selling the lottery, i.e.,
	\begin{equation}
		\label{eq_2_1_main}
		u\left(w+S^*\right)=(1-p)\cdot u(w+B)+p\cdot u(w+G)
	\end{equation}
	This equation determines the minimum price $S^{*}$.
\subsection*{2)}
If the individual does not own the lottery, suppose the price he would buy it for is $P$. Then the expected utility if he buys the lottery would be:
	\begin{equation}
		u=(1-p)\cdot u(w+B-P)+p\cdot u(w+G-P)
	\end{equation}
	On the other hand, if he does not buy the lottery, the expected utility would be:
	\begin{equation}
		u = u (w)
	\end{equation}
	The maximum price he would buy the lottery for, $P^{*}$, is the price which makes the individual indifferent about buying the lottery, i.e.,
	\begin{equation}
		\label{eq_2_2_main}
		u(w)=(1-p)\cdot u\left(w+B-P^*\right)+p\cdot u\left(w+G-P^*\right)
	\end{equation}
	This equation determines the maximum price $P^{*}$.
\subsection*{3)}
From \eqref{eq_2_1_main} and \eqref{eq_2_2_main} we can see that generally speaking, we \textbf{don't} have $S^*=P^*$, i.e. the minimum price the individual would sell the lottery for if he owns it \textbf{does not equate} the maximum price he would buy it for if he does not own it.
	\par
	Compare \eqref{eq_2_1_main} with \eqref{eq_2_2_main}, we can get:
	\begin{equation}
		\label{eq_2_3_P_and_S_relation}
		\begin{aligned}
			P^{*} (w) & = S^{*}(w-P^{*}(w)) \\
			S^{*} (w) & = P^{*} (w + S^{*}(w)) 
		\end{aligned}
	\end{equation}
	Hence, we have:
	\begin{equation}
		\label{eq_2_3_diff_S_and_P}
		S^{*} (w) - P^{*}(w) = S^{*} (w) -  S^{*}(w-P^{*}(w))
	\end{equation}
	The economic implication of \eqref{eq_2_3_P_and_S_relation} and \eqref{eq_2_3_diff_S_and_P} is evident: the reason that $S^{*}(w) \neq P^{*}(w)$ is that the individual actually has different total wealth in the two cases: when the individual owns the lottery, his effective wealth is $w + S^{*}(w)$, but when the individual does not own the lottery, his effective wealth is $w$. If the individual has different risk preference under different total wealth, then he would assign different values to the lottery.
	\par
	From the analysis above, we can conclude that if the individual has the same risk preference under different total wealth, then we would have 
	\begin{equation}
		S^{*} (w) =  S^{*}(w-P^{*}(w))
	\end{equation}
	which is equivalent to
	\begin{equation}
		S^{*} (w) = P^{*}(w)
	\end{equation}
	Specifically, if 
	\begin{equation}
		u(w) = - a e^{-\lambda w}
	\end{equation}
	then we can easily verify that 
	\begin{equation}
		S^{*} = P^{*}
	\end{equation}

\section*{Q3}
\subsection*{1)}
The individual faces the following optimization problem:
	\begin{equation}
		\underset{x}{\text{max: }}\nu(w-x) + \nu(x)
	\end{equation}
	Since the function $\nu(.)$ is concave, we have:
	\begin{equation}
		\frac{1}{2} \left[ \nu(w-x)+ \nu(x) \right] \leq \nu\left( \frac{w}{2} \right)
	\end{equation}
	Thus the optimal saving $x_0 = w/2$.
\subsection*{2)}
Now the optimization problem becomes:
	\begin{equation}
		\underset{x}{\text{max: }}\nu(w-x) + E[\nu(x+y)]
	\end{equation}
	
	Differentiate with respect to $x$: 
	\begin{equation}
		u^{\prime}(x) = E\left[ \nu^{\prime}(x+y)\right] -\nu^{\prime}(w-x)
	\end{equation}
	Differentiate again:
	\begin{equation}
		\label{eq_3_2_mu_diff}
		u^{\prime\prime}(x) = E\left[ \nu^{\prime\prime}(x+y)\right] + \nu^{\prime\prime}(w-x)
	\end{equation}
	Since $\nu(.)$ is concave, we have $\nu^{\prime\prime}(.) < 0$. Thus from \eqref{eq_3_2_mu_diff} we can see that 
	\begin{equation}
		u^{\prime\prime}(x) < 0
	\end{equation}
	i.e., $u^{\prime}(.)$ is a decreasing function. Thus we can simply let $u^{\prime}(x^{*})=0$ to get $x^{*}$, i.e.,
	\begin{equation}
		\label{eq_3_2_def_x_star}
		-\nu^{\prime}(w-x^{*}) + E\left[\nu^{\prime}(x^{*}+y)\right] = 0
	\end{equation}
	This is the equation that determines $x^{*}$.
\subsection*{3)} 
If $E\left[\nu^{\prime}(x_0+y)\right]>\nu^{\prime}(x_0)$, it implies that 
	\begin{equation}
		u^{\prime}(x_0) = E\left[ \nu^{\prime}(x_0+y)\right] -\nu^{\prime}(w-x_0) = E\left[ \nu^{\prime}(x_0+y)\right] -\nu^{\prime}(x_0) > 0
	\end{equation}
	The last step uses the fact that $x_0 = w/2$. Also, we have 
	\begin{equation}
		u^{\prime}(x^{*}) = 0
	\end{equation}
	Hence,  
	\begin{equation}
		u^{\prime}(x_0) > u^{\prime}(x^{*})
	\end{equation}
	We have proved that $u^{\prime}(.)$ is a decreasing function, thus, 
	\begin{equation}
		x^{*} > x_0
	\end{equation}

	\section*{Q4}
	\subsection*{1)} The definition of the risk premium for $\tilde{\varepsilon}$ is
\begin{equation}
	E\left[ u\left(\tilde{w}-\pi _u\right) \right] = E \left[  u\left(\tilde{w} + \tilde{\varepsilon}\right) \right]
\end{equation}
Expand the expected value:
\begin{equation}
	\label{eq_4_def_risk_premium}
	p \,u(w_1 - \pi_u) + (1-p)\, u(w_2 - \pi_u) = p \left[ \frac{1}{2} u(w_1 + \varepsilon) + \frac{1}{2}u(w_1-\varepsilon)\right] + (1-p) \,u(w_2)
\end{equation}
If $\varepsilon$ is sufficiently small, we can expect $\pi_u$ is also small. We expand the LHS of \eqref{eq_4_def_risk_premium} to the smallest order of $\pi_u$ possible (at least to the first order):
\begin{equation}
	\label{eq_4_taylor_def_LHS}
	\begin{aligned}
		& p\, u(w_1 - \pi_u) + (1-p)\, u(w_2 - \pi_u) = \\
		& p \,u(w_1) + (1-p)\, u(w_2) - \left[p \,u^{\prime}(w_1) + (1-p)\, u^{\prime}(w_2)\right] \pi_u + O\left( \pi_u^{2} \right)
	\end{aligned}
\end{equation}
Expand the RHS of \eqref{eq_4_def_risk_premium} to the smallest order of $\varepsilon$ possible (at least to the first order):
\begin{equation}
	\label{eq_4-taylor_def_RHS}
	\begin{aligned}
		& p \left[ \frac{1}{2} u(w_1 + \varepsilon) + \frac{1}{2}u(w_1-\varepsilon)\right] + (1-p)\, u(w_2) = \\
		& p \left[ u(w_1) + \frac{1}{2}u^{\prime\prime}(w_1)\varepsilon^2 \right] + (1-p)\, u(w_2) + O(\varepsilon^4)
	\end{aligned}
\end{equation}
Compare \eqref{eq_4_taylor_def_LHS} and \eqref{eq_4-taylor_def_RHS} we have:
\begin{equation}
	- \left[p\, u^{\prime}(w_1) + (1-p)\, u^{\prime}(w_2)\right] \pi_u \approx  \frac{1}{2}p\,u^{\prime\prime}(w_1)\varepsilon^2
\end{equation}
which leads to
\begin{equation}
	\pi_u \approx -\frac{\frac{1}{2}p\,u^{\prime\prime}(w_1)\varepsilon^2}{p\, u^{\prime}(w_1) + (1-p)\, u^{\prime}(w_2)}
\end{equation}
$\blacksquare$
	\subsection*{2)} The Arrow-Pratt measure is defined as:
	\begin{equation}
		A(w) = -\frac{u^{\prime\prime}(w)}{u^{\prime}(w)}
	\end{equation}
	For utility function that has the form
	\begin{equation}
		\begin{aligned}
			u(w) & = -e^{-a\,w} \\
			v(w) & = -e^{-b\,w}
		\end{aligned}
	\end{equation}
	the Arrow-Pratt measure is:
	\begin{equation}
		\begin{aligned}
			A_{u}(w) & = a \\
			A_{v}(w) & = b
		\end{aligned}
	\end{equation}
	\subsection*{3)} Plug the utility function:
	\begin{equation}
		u(\lambda,w) = - e^{-\lambda\,w}
	\end{equation}
	into \eqref{eq_4_def_risk_premium}, we get:
	\begin{equation}
		p e^{\lambda \pi} + (1-p) e^{\lambda \pi} e^{\lambda (w_1 - w_2)} = p \left( \frac{1}{2} e^{\lambda \varepsilon} + \frac{1}{2} e^{-\lambda \varepsilon} \right) + (1-p) e^{\lambda(w_1 - w_2)}
	\end{equation}
	Thus we have:
	\begin{equation}
		e^{\lambda \pi} = \frac{p \left( \frac{1}{2} e^{\lambda \varepsilon} + \frac{1}{2} e^{-\lambda \varepsilon} \right) + (1-p) e^{\lambda(w_1 - w_2)}}{p + (1-p) e^{\lambda (w_1 - w_2)} }
	\end{equation}
	which is equivalent to
	\begin{equation}
		e^{\lambda \pi} = 1+ \frac{p \left( \frac{1}{2} e^{\lambda \varepsilon} + \frac{1}{2} e^{-\lambda \varepsilon} -1 \right)}{p + (1-p) e^{\lambda (w_1 - w_2)} }
	\end{equation}
	We thus have:
	\begin{equation}
		\label{eq_4_risk_premium_calculation}
		\pi = \frac{1}{\lambda} \log \left[ 1+ \frac{p \left( \frac{1}{2} e^{\lambda \varepsilon} + \frac{1}{2} e^{-\lambda \varepsilon} -1 \right)}{p + (1-p) e^{\lambda (w_1 - w_2)} } \right]
	\end{equation}
	Consider the numerator and denominator of 
	\begin{equation}
		\frac{p \left( \frac{1}{2} e^{\lambda \varepsilon} + \frac{1}{2} e^{-\lambda \varepsilon} -1 \right)}{p + (1-p) e^{\lambda (w_1 - w_2)} }
	\end{equation}
	We have:
	\begin{equation}
		\begin{aligned}
			 \frac{\partial}{\partial \lambda} \left[ p \left( \frac{1}{2} e^{\lambda \varepsilon} + \frac{1}{2} e^{-\lambda \varepsilon} -1 \right) \right] & =  p \left( \frac{\varepsilon}{2} e^{\lambda \varepsilon} - \frac{\varepsilon}{2} e^{-\lambda \varepsilon} \right) \\
			 \frac{\partial}{\partial \lambda} \left[ p + (1-p) e^{\lambda (w_1 - w_2)} \right] & = 
			 (1-p)(w_1 - w_2)e^{\lambda (w_1 - w_2)} 
		\end{aligned}
	\end{equation}
	Now consider $a>b$. If $w_1 - w_2$ is large enough s.t.:
	\begin{equation}
		w_1 - w_2 > \max_{\lambda \in [b,a]} \left[ \frac{\frac{\varepsilon}{2}e^{\lambda\varepsilon}}{(1-p)\left( \frac{1}{2} e^{\lambda \varepsilon} + \frac{1}{2} e^{-\lambda \varepsilon} -1 \right)} \right]
	\end{equation}
	Then, for $\lambda \in [b,a]$ we have:
	\begin{equation}
		\frac{p \left( \frac{\varepsilon}{2} e^{\lambda \varepsilon} - \frac{\varepsilon}{2} e^{-\lambda \varepsilon} \right) }{(1-p)(w_1 - w_2)e^{\lambda (w_1 - w_2)} }<\frac{p \left( \frac{1}{2} e^{\lambda \varepsilon} + \frac{1}{2} e^{-\lambda \varepsilon} -1 \right)}{p + (1-p) e^{\lambda (w_1 - w_2)} }
	\end{equation}
	which is
	\begin{equation}
		\frac{\frac{\partial}{\partial \lambda} \left[ p \left( \frac{1}{2} e^{\lambda \varepsilon} + \frac{1}{2} e^{-\lambda \varepsilon} -1 \right) \right]}{\frac{\partial}{\partial \lambda} \left[ p + (1-p) e^{\lambda (w_1 - w_2)} \right]} < \frac{p \left( \frac{1}{2} e^{\lambda \varepsilon} + \frac{1}{2} e^{-\lambda \varepsilon} -1 \right)}{p + (1-p) e^{\lambda (w_1 - w_2)} }
	\end{equation}
	Thus,  
	\begin{equation}
		\frac{p \left( \frac{1}{2} e^{\lambda \varepsilon} + \frac{1}{2} e^{-\lambda \varepsilon} -1 \right)}{p + (1-p) e^{\lambda (w_1 - w_2)} }
	\end{equation}
	is a decreasing function of $\lambda$ on $[b,a]$. Hence we must have:
	\begin{equation}
		\pi(b) > \pi(a)
	\end{equation}
	that is,
	\begin{equation}
		\pi_v > \pi_u
	\end{equation}

	\section*{Q5}
	Now the risk premium is defined as
\begin{equation}
	\frac{1}{2}u(w+\pi+\sigma) + \frac{1}{2}u(w+\pi-\sigma) = u(w)
\end{equation}
Suppose $\sigma$ is small enough, we can thus expect that $\pi$ is also small. It is easy to notice that $\sigma$ vanishes in the first order Taylor expansion of the LHS of the definition above. Thus, we expand to the second order:
\begin{equation}
	u(w) + u^{\prime}(w) \pi + \frac{1}{4}u^{\prime\prime}(w)\left[ (\pi+\sigma)^2 + (\pi-\sigma)^2 \right] + O((\pi\pm \sigma)^3) = u(w)
\end{equation}
This can be written as:
\begin{equation}
	u^{\prime}(w) \pi + \frac{1}{2}u^{\prime\prime}(w)\pi^2 + \frac{1}{2}u^{\prime\prime}(w)\sigma^2 + O((\pi \pm \sigma)^3) = 0
\end{equation}
Note that $\pi =O(\sigma ^2)$, thus $\pi^2$ term above can be dropped, and we have:
\begin{equation}
	u^{\prime}(w) \pi  + \frac{1}{2}u^{\prime\prime}(w)\sigma^2 \approx 0
\end{equation}
Thus we have
\begin{equation}
	\pi \approx -\frac{u^{\prime\prime}(w)}{2 u^{\prime}(w)}\sigma^2
\end{equation}
$\blacksquare$
% \begin{appendix}
% \section*{Appendix}
% \lstinputlisting[language=mathematica]{Q4plot.nb}
% \end{appendix}
\end{document}

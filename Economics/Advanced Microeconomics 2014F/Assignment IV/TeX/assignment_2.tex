%%%%%%%%%%%%%%%%%%%%%%%%%%%%%%%%%%%%%%%%%
% Programming/Coding Assignment
% LaTeX Template
%
% This template has been downloaded from:
% http://www.latextemplates.com
%
% Original author:
% Ted Pavlic (http://www.tedpavlic.com)
%
% Note:
% The \lipsum[#] commands throughout this template generate dummy text
% to fill the template out. These commands should all be removed when 
% writing assignment content.
%
% This template uses a Perl script as an example snippet of code, most other
% languages are also usable. Configure them in the "CODE INCLUSION 
% CONFIGURATION" section.
%
%%%%%%%%%%%%%%%%%%%%%%%%%%%%%%%%%%%%%%%%%

%----------------------------------------------------------------------------------------
%	PACKAGES AND OTHER DOCUMENT CONFIGURATIONS
%----------------------------------------------------------------------------------------

\documentclass{article}

\usepackage{fancyhdr} % Required for custom headers
\usepackage{lastpage} % Required to determine the last page for the footer
\usepackage{extramarks} % Required for headers and footers
\usepackage[usenames,dvipsnames]{color} % Required for custom colors
\usepackage{graphicx} % Required to insert images
\usepackage{listings} % Required for insertion of code
\usepackage{courier} % Required for the courier font
\usepackage{lipsum} % Used for inserting dummy 'Lorem ipsum' text into the template
\usepackage{amsmath}

% Margins
\topmargin=-0.45in
\evensidemargin=0in
\oddsidemargin=0in
\textwidth=6.5in
\textheight=9.0in
\headsep=0.25in

\linespread{1.1} % Line spacing

% Set up the header and footer
\pagestyle{fancy}
\lhead{\hmwkAuthorName} % Top left header
\chead{\hmwkClass\ : \hmwkTitle} % Top center head
%\rhead{\firstxmark} % Top right header
%\lfoot{\lastxmark} % Bottom left footer
\cfoot{} % Bottom center footer
\rfoot{Page\ \thepage\ of\ \protect\pageref{LastPage}} % Bottom right footer
\renewcommand\headrulewidth{0.4pt} % Size of the header rule
\renewcommand\footrulewidth{0.4pt} % Size of the footer rule

\setlength\parindent{0pt} % Removes all indentation from paragraphs

%----------------------------------------------------------------------------------------
%	CODE INCLUSION CONFIGURATION
%----------------------------------------------------------------------------------------

\definecolor{MyDarkGreen}{rgb}{0.0,0.4,0.0} % This is the color used for comments
\lstloadlanguages{Perl} % Load Perl syntax for listings, for a list of other languages supported see: ftp://ftp.tex.ac.uk/tex-archive/macros/latex/contrib/listings/listings.pdf
\lstset{language=Perl, % Use Perl in this example
        frame=single, % Single frame around code
        basicstyle=\small\ttfamily, % Use small true type font
        keywordstyle=[1]\color{Blue}\bf, % Perl functions bold and blue
        keywordstyle=[2]\color{Purple}, % Perl function arguments purple
        keywordstyle=[3]\color{Blue}\underbar, % Custom functions underlined and blue
        identifierstyle=, % Nothing special about identifiers                                         
        commentstyle=\usefont{T1}{pcr}{m}{sl}\color{MyDarkGreen}\small, % Comments small dark green courier font
        stringstyle=\color{Purple}, % Strings are purple
        showstringspaces=false, % Don't put marks in string spaces
        tabsize=5, % 5 spaces per tab
        %
        % Put standard Perl functions not included in the default language here
        morekeywords={rand},
        %
        % Put Perl function parameters here
        morekeywords=[2]{on, off, interp},
        %
        % Put user defined functions here
        morekeywords=[3]{test},
       	%
        morecomment=[l][\color{Blue}]{...}, % Line continuation (...) like blue comment
        numbers=left, % Line numbers on left
        firstnumber=1, % Line numbers start with line 1
        numberstyle=\tiny\color{Blue}, % Line numbers are blue and small
        stepnumber=5 % Line numbers go in steps of 5
}

% Creates a new command to include a perl script, the first parameter is the filename of the script (without .pl), the second parameter is the caption
\newcommand{\perlscript}[2]{
\begin{itemize}
\item[]\lstinputlisting[caption=#2,label=#1]{#1.pl}
\end{itemize}
}

%----------------------------------------------------------------------------------------
%	DOCUMENT STRUCTURE COMMANDS
%	Skip this unless you know what you're doing
%----------------------------------------------------------------------------------------

% Header and footer for when a page split occurs within a problem environment
\newcommand{\enterProblemHeader}[1]{
\nobreak\extramarks{#1}{#1 continued on next page\ldots}\nobreak
\nobreak\extramarks{#1 (continued)}{#1 continued on next page\ldots}\nobreak
}

% Header and footer for when a page split occurs between problem environments
\newcommand{\exitProblemHeader}[1]{
\nobreak\extramarks{#1 (continued)}{#1 continued on next page\ldots}\nobreak
\nobreak\extramarks{#1}{}\nobreak
}

\setcounter{secnumdepth}{0} % Removes default section numbers
\newcounter{homeworkProblemCounter} % Creates a counter to keep track of the number of problems

\newcommand{\homeworkProblemName}{}
\newenvironment{homeworkProblem}[1][Problem \arabic{homeworkProblemCounter}]{ % Makes a new environment called homeworkProblem which takes 1 argument (custom name) but the default is "Problem #"
\stepcounter{homeworkProblemCounter} % Increase counter for number of problems
\renewcommand{\homeworkProblemName}{#1} % Assign \homeworkProblemName the name of the problem
\section{\homeworkProblemName} % Make a section in the document with the custom problem count
\enterProblemHeader{\homeworkProblemName} % Header and footer within the environment
}{
\exitProblemHeader{\homeworkProblemName} % Header and footer after the environment
}

\newcommand{\problemAnswer}[1]{ % Defines the problem answer command with the content as the only argument
\noindent\framebox[\columnwidth][c]{\begin{minipage}{0.98\columnwidth}#1\end{minipage}} % Makes the box around the problem answer and puts the content inside
}

\newcommand{\homeworkSectionName}{}
\newenvironment{homeworkSection}[1]{ % New environment for sections within homework problems, takes 1 argument - the name of the section
\renewcommand{\homeworkSectionName}{#1} % Assign \homeworkSectionName to the name of the section from the environment argument
\subsection{\homeworkSectionName} % Make a subsection with the custom name of the subsection
\enterProblemHeader{\homeworkProblemName\ [\homeworkSectionName]} % Header and footer within the environment
}{
\enterProblemHeader{\homeworkProblemName} % Header and footer after the environment
}

%----------------------------------------------------------------------------------------
%	NAME AND CLASS SECTION
%----------------------------------------------------------------------------------------

\newcommand{\hmwkTitle}{Assignment\ 4} % Assignment title
\newcommand{\hmwkDueDate}{Tuesday,\ December\ 16,\ 2014} % Due date
\newcommand{\hmwkClass}{Advanced Microeconomics} % Course/class
\newcommand{\hmwkClassTime}{} % Class/lecture time
\newcommand{\hmwkClassInstructor}{} % Teacher/lecturer
\newcommand{\hmwkAuthorName}{Xie Yuchen 1401211838} % Your name

%----------------------------------------------------------------------------------------
%	TITLE PAGE
%----------------------------------------------------------------------------------------

\title{
\vspace{2in}
\textmd{\textbf{\hmwkClass:\ \hmwkTitle}}\\
\normalsize\vspace{0.1in}\small{Due\ on\ \hmwkDueDate}\\
\vspace{0.1in}\large{\textit{\hmwkClassInstructor\ \hmwkClassTime}}
\vspace{3in}
}

\author{\textbf{\hmwkAuthorName}}
\date{} % Insert date here if you want it to appear below your name

%----------------------------------------------------------------------------------------

\begin{document}

\maketitle

%----------------------------------------------------------------------------------------
%	TABLE OF CONTENTS
%----------------------------------------------------------------------------------------

%\setcounter{tocdepth}{1} % Uncomment this line if you don't want subsections listed in the ToC

\newpage
\tableofcontents
\newpage

%----------------------------------------------------------------------------------------
%	PROBLEM 1
%----------------------------------------------------------------------------------------

% To have just one problem per page, simply put a \clearpage after each problembeg
\begin{homeworkProblem}
Suppose the individual's total amount of asset is $A$, and his allocation between two assets is $\left(A_1,A_2\right)$, where $A_1+A_2=A$. The individual maximizes his expected utility level, which is
\begin{equation}
	E\left[u\left(A_1 R_1+A_2 R_2+A\right)\right]
\end{equation}
First let's suppose the individual is risk averse, i.e.
\begin{equation}
	u^{\prime\prime}(w) < 0
\end{equation}
Since $R_1$ and $R_2$ satisfy the same distribution (with CDF $f(.)$), the expected utility can be written as:
\begin{equation}
	\label{eq_1_1_int_trans}
	\begin{aligned}
		E\left[u\left(A_1 R_1+A_2 R_2+A\right)\right] & = E\left[u\left(A_2 R_1+A_1 R_2+A\right)\right] \\
		& = \frac{1}{2} \left( E\left[u\left(A_1 R_1+A_2 R_2+A\right)\right] +  E\left[u\left(A_2 R_1+A_1 R_2+A\right)\right] \right) \\
		& = E \left( \frac{1}{2} \left[ u(A+A_1 R_1 + A_2 R_2) + u(A+A_2 R_1 + A_1 R_2) \right] \right)
	\end{aligned}
\end{equation}
Note that the individual is risk averse, hence $u^{\prime\prime}(w) < 0$, we must have
\begin{equation}
	\label{eq_1_1_neq}
	\frac{1}{2} \left[ u(A+A_1 R_1 + A_2 R_2) + u(A+A_2 R_1 + A_1 R_2) \right] \leq u \left( A+\frac{1}{2}A R_1 + \frac{1}{2}A R_2 \right)
\end{equation}
This simply follows from the property of concave functions and the fact that $A_1 + A_2 = A$. Note \eqref{eq_1_1_neq} holds for any allocation $(A_1, A_2)$ and return $(R_1, R_2)$. Thus, from \eqref{eq_1_1_int_trans} and \eqref{eq_1_1_neq}, we must have:
\begin{equation}
	E\left[u\left(A_1 R_1+A_2 R_2+A\right)\right] \leq E\left[u\left(\frac{1}{2} A R_1+  \frac{1}{2} A R_2+A\right)\right]
\end{equation}
 Thus we can conclude that if the individual is risk averse, then he maximize his expected utility when he evenly allocates his money between two assets.
\par
Now let's suppose the individual is risk loving. It then follows that
\begin{equation}
	u^{\prime\prime}(w) > 0
\end{equation}
From the property of convex functions, we have:
\begin{equation}
	\frac{A_1}{A}u(A+A R_1) + \frac{A_2}{A}u(A+A R_2) \geq u(A+A_1 R_1 + A_2 R_2)
\end{equation}
This also holds for any allocation $(A_1, A_2)$ and return $(R_1, R_2)$. Hence, we have:
\begin{equation}
	\label{eq_1_2_neq}
	\frac{A_1}{A}E\left[ u(A+A R_1) \right] + \frac{A_2}{A} E \left[ u(A+A R_2) \right] \geq E \left[ u(A+A_1 R_1 + A_2 R_2) \right]
\end{equation}
Since $R_1$ and $R_2$ satisfy the same distribution, we have:
\begin{equation}
	E\left[ u(A+A R_1) \right] = E \left[ u(A+A R_2) \right]
\end{equation}
Thus \eqref{eq_1_2_neq} can be written as
\begin{equation}
	E\left[ u(A+A R_1) \right] \geq E \left[ u(A+A_1 R_1 + A_2 R_2) \right]
\end{equation}
This holds for any allocation $(A_1, A_2)$. Thus, when the individual is risk loving, the expected utility is maximized when the individual allocates all his money on one asset.
\end{homeworkProblem}

\newpage

\begin{homeworkProblem}
\begin{homeworkSection}{(1)}
	If the individual owns the lottery, suppose the price he would sell it for is $S$. Then the expected utility if he sells the lottery would be:
	\begin{equation}
		u = u (w+S)
	\end{equation}
	On the other hand, if he does not sell the lottery, the expected utility would be:
	\begin{equation}
		u = p \cdot u(w+G) + (1-p) \cdot u(w+B)
	\end{equation}
	The minimum price he would sell the lottery for, $S^{*}$, is the price which makes the individual indifferent about selling the lottery, i.e.,
	\begin{equation}
		\label{eq_2_1_main}
		u\left(w+S^*\right)=(1-p)\cdot u(w+B)+p\cdot u(w+G)
	\end{equation}
	This equation determines the minimum price $S^{*}$.
\end{homeworkSection}
\begin{homeworkSection}{(2)}
	If the individual does not own the lottery, suppose the price he would buy it for is $P$. Then the expected utility if he buys the lottery would be:
	\begin{equation}
		u=(1-p)\cdot u(w+B-P)+p\cdot u(w+G-P)
	\end{equation}
	On the other hand, if he does not buy the lottery, the expected utility would be:
	\begin{equation}
		u = u (w)
	\end{equation}
	The maximum price he would buy the lottery for, $P^{*}$, is the price which makes the individual indifferent about buying the lottery, i.e.,
	\begin{equation}
		\label{eq_2_2_main}
		u(w)=(1-p)\cdot u\left(w+B-P^*\right)+p\cdot u\left(w+G-P^*\right)
	\end{equation}
	This equation determines the maximum price $P^{*}$.
\end{homeworkSection}
\begin{homeworkSection}{(3)}
	From \eqref{eq_2_1_main} and \eqref{eq_2_2_main} we can see that generally speaking, we \textbf{don't} have $S^*=P^*$, i.e. the minimum price the individual would sell the lottery for if he owns it \textbf{does not equate} the maximum price he would buy it for if he does not own it.
	\par
	Compare \eqref{eq_2_1_main} with \eqref{eq_2_2_main}, we can get:
	\begin{equation}
		\label{eq_2_3_P_and_S_relation}
		\begin{aligned}
			P^{*} (w) & = S^{*}(w-P^{*}(w)) \\
			S^{*} (w) & = P^{*} (w + S^{*}(w)) 
		\end{aligned}
	\end{equation}
	Hence, we have:
	\begin{equation}
		\label{eq_2_3_diff_S_and_P}
		S^{*} (w) - P^{*}(w) = S^{*} (w) -  S^{*}(w-P^{*}(w))
	\end{equation}
	The economic implication of \eqref{eq_2_3_P_and_S_relation} and \eqref{eq_2_3_diff_S_and_P} is evident: the reason that $S^{*}(w) \neq P^{*}(w)$ is that the individual actually has different total wealth in the two cases: when the individual owns the lottery, his effective wealth is $w + S^{*}(w)$, but when the individual does not own the lottery, his effective wealth is $w$. If the individual has different risk preference under different total wealth, then he would assign different values to the lottery.
	\par
	From the analysis above, we can conclude that if the individual has the same risk preference under different total wealth, then we would have 
	\begin{equation}
		S^{*} (w) =  S^{*}(w-P^{*}(w))
	\end{equation}
	which is equivalent to
	\begin{equation}
		S^{*} (w) = P^{*}(w)
	\end{equation}
	Specifically, if 
	\begin{equation}
		u(w) = - a e^{-\lambda w}
	\end{equation}
	then we can easily verify that 
	\begin{equation}
		S^{*} = P^{*}
	\end{equation}
	
\end{homeworkSection}
\end{homeworkProblem}

\newpage

\begin{homeworkProblem}
\begin{homeworkSection}{(1)}
	The individual faces the following optimization problem:
	\begin{equation}
		\underset{x}{\text{max: }}\nu(w-x) + \nu(x)
	\end{equation}
	Since the function $\nu(.)$ is concave, we have:
	\begin{equation}
		\frac{1}{2} \left[ \nu(w-x)+ \nu(x) \right] \leq \nu\left( \frac{w}{2} \right)
	\end{equation}
	Thus the optimal saving $x_0 = w/2$.
\end{homeworkSection}

\begin{homeworkSection}{(2)}
	Now the optimization problem becomes:
	\begin{equation}
		\underset{x}{\text{max: }}\nu(w-x) + E[\nu(x+y)]
	\end{equation}
	
	Differentiate with respect to $x$: 
	\begin{equation}
		u^{\prime}(x) = E\left[ \nu^{\prime}(x+y)\right] -\nu^{\prime}(w-x)
	\end{equation}
	Differentiate again:
	\begin{equation}
		\label{eq_3_2_mu_diff}
		u^{\prime\prime}(x) = E\left[ \nu^{\prime\prime}(x+y)\right] + \nu^{\prime\prime}(w-x)
	\end{equation}
	Since $\nu(.)$ is concave, we have $\nu^{\prime\prime}(.) < 0$. Thus from \eqref{eq_3_2_mu_diff} we can see that 
	\begin{equation}
		u^{\prime\prime}(x) < 0
	\end{equation}
	i.e., $u^{\prime}(.)$ is a decreasing function. Thus we can simply let $u^{\prime}(x^{*})=0$ to get $x^{*}$, i.e.,
	\begin{equation}
		\label{eq_3_2_def_x_star}
		-\nu^{\prime}(w-x^{*}) + E\left[\nu^{\prime}(x^{*}+y)\right] = 0
	\end{equation}
	This is the equation that determines $x^{*}$.
\end{homeworkSection}

\begin{homeworkSection}{(3)}
 If $E\left[\nu^{\prime}(x_0+y)\right]>\nu^{\prime}(x_0)$, it implies that 
	\begin{equation}
		u^{\prime}(x_0) = E\left[ \nu^{\prime}(x_0+y)\right] -\nu^{\prime}(w-x_0) = E\left[ \nu^{\prime}(x_0+y)\right] -\nu^{\prime}(x_0) > 0
	\end{equation}
	The last step uses the fact that $x_0 = w/2$. Also, we have 
	\begin{equation}
		u^{\prime}(x^{*}) = 0
	\end{equation}
	Hence,  
	\begin{equation}
		u^{\prime}(x_0) > u^{\prime}(x^{*})
	\end{equation}
	We have proved that $u^{\prime}(.)$ is a decreasing function, thus, 
	\begin{equation}
		x^{*} > x_0
	\end{equation}
\end{homeworkSection}
\end{homeworkProblem}

\newpage

\begin{homeworkProblem}
\begin{homeworkSection}{(1)}
The definition of the risk premium for $\tilde{\varepsilon}$ is
\begin{equation}
	E\left[ u\left(\tilde{w}-\pi _u\right) \right] = E \left[  u\left(\tilde{w} + \tilde{\varepsilon}\right) \right]
\end{equation}
Expand the expected value:
\begin{equation}
	\label{eq_4_def_risk_premium}
	p \,u(w_1 - \pi_u) + (1-p)\, u(w_2 - \pi_u) = p \left[ \frac{1}{2} u(w_1 + \varepsilon) + \frac{1}{2}u(w_1-\varepsilon)\right] + (1-p) \,u(w_2)
\end{equation}
If $\varepsilon$ is sufficiently small, we can expect $\pi_u$ is also small. We expand the LHS of \eqref{eq_4_def_risk_premium} to the smallest order of $\pi_u$ possible (at least to the first order):
\begin{equation}
	\label{eq_4_taylor_def_LHS}
	\begin{aligned}
		& p\, u(w_1 - \pi_u) + (1-p)\, u(w_2 - \pi_u) = \\
		& p \,u(w_1) + (1-p)\, u(w_2) - \left[p \,u^{\prime}(w_1) + (1-p)\, u^{\prime}(w_2)\right] \pi_u + O\left( \pi_u^{2} \right)
	\end{aligned}
\end{equation}
Expand the RHS of \eqref{eq_4_def_risk_premium} to the smallest order of $\varepsilon$ possible (at least to the first order):
\begin{equation}
	\label{eq_4-taylor_def_RHS}
	\begin{aligned}
		& p \left[ \frac{1}{2} u(w_1 + \varepsilon) + \frac{1}{2}u(w_1-\varepsilon)\right] + (1-p)\, u(w_2) = \\
		& p \left[ u(w_1) + \frac{1}{2}u^{\prime\prime}(w_1)\varepsilon^2 \right] + (1-p)\, u(w_2) + O(\varepsilon^4)
	\end{aligned}
\end{equation}
Compare \eqref{eq_4_taylor_def_LHS} and \eqref{eq_4-taylor_def_RHS} we have:
\begin{equation}
	- \left[p\, u^{\prime}(w_1) + (1-p)\, u^{\prime}(w_2)\right] \pi_u \approx  \frac{1}{2}p\,u^{\prime\prime}(w_1)\varepsilon^2
\end{equation}
which leads to
\begin{equation}
	\pi_u \approx -\frac{\frac{1}{2}p\,u^{\prime\prime}(w_1)\varepsilon^2}{p\, u^{\prime}(w_1) + (1-p)\, u^{\prime}(w_2)}
\end{equation}
\end{homeworkSection}
\begin{homeworkSection}{(2)}
	The Arrow-Pratt measure is defined as:
	\begin{equation}
		A(w) = -\frac{u^{\prime\prime}(w)}{u^{\prime}(w)}
	\end{equation}
	For utility function that has the form
	\begin{equation}
		\begin{aligned}
			u(w) & = -e^{-a\,w} \\
			v(w) & = -e^{-b\,w}
		\end{aligned}
	\end{equation}
	the Arrow-Pratt measure is:
	\begin{equation}
		\begin{aligned}
			A_{u}(w) & = a \\
			A_{v}(w) & = b
		\end{aligned}
	\end{equation}
\end{homeworkSection}
\begin{homeworkSection}{(3)}
	Plug the utility function:
	\begin{equation}
		u(\lambda,w) = - e^{-\lambda\,w}
	\end{equation}
	into \eqref{eq_4_def_risk_premium}, we get:
	\begin{equation}
		p e^{\lambda \pi} + (1-p) e^{\lambda \pi} e^{\lambda (w_1 - w_2)} = p \left( \frac{1}{2} e^{\lambda \varepsilon} + \frac{1}{2} e^{-\lambda \varepsilon} \right) + (1-p) e^{\lambda(w_1 - w_2)}
	\end{equation}
	Thus we have:
	\begin{equation}
		e^{\lambda \pi} = \frac{p \left( \frac{1}{2} e^{\lambda \varepsilon} + \frac{1}{2} e^{-\lambda \varepsilon} \right) + (1-p) e^{\lambda(w_1 - w_2)}}{p + (1-p) e^{\lambda (w_1 - w_2)} }
	\end{equation}
	which is equivalent to
	\begin{equation}
		e^{\lambda \pi} = 1+ \frac{p \left( \frac{1}{2} e^{\lambda \varepsilon} + \frac{1}{2} e^{-\lambda \varepsilon} -1 \right)}{p + (1-p) e^{\lambda (w_1 - w_2)} }
	\end{equation}
	We thus have:
	\begin{equation}
		\label{eq_4_risk_premium_calculation}
		\pi = \frac{1}{\lambda} \log \left[ 1+ \frac{p \left( \frac{1}{2} e^{\lambda \varepsilon} + \frac{1}{2} e^{-\lambda \varepsilon} -1 \right)}{p + (1-p) e^{\lambda (w_1 - w_2)} } \right]
	\end{equation}
	Consider the numerator and denominator of 
	\begin{equation}
		\frac{p \left( \frac{1}{2} e^{\lambda \varepsilon} + \frac{1}{2} e^{-\lambda \varepsilon} -1 \right)}{p + (1-p) e^{\lambda (w_1 - w_2)} }
	\end{equation}
	We have:
	\begin{equation}
		\begin{aligned}
			 \frac{\partial}{\partial \lambda} \left[ p \left( \frac{1}{2} e^{\lambda \varepsilon} + \frac{1}{2} e^{-\lambda \varepsilon} -1 \right) \right] & =  p \left( \frac{\varepsilon}{2} e^{\lambda \varepsilon} - \frac{\varepsilon}{2} e^{-\lambda \varepsilon} \right) \\
			 \frac{\partial}{\partial \lambda} \left[ p + (1-p) e^{\lambda (w_1 - w_2)} \right] & = 
			 (1-p)(w_1 - w_2)e^{\lambda (w_1 - w_2)} 
		\end{aligned}
	\end{equation}
	Now consider $a>b$. If $w_1 - w_2$ is large enough s.t.:
	\begin{equation}
		w_1 - w_2 > \max_{\lambda \in [b,a]} \left[ \frac{\frac{\varepsilon}{2}e^{\lambda\varepsilon}}{(1-p)\left( \frac{1}{2} e^{\lambda \varepsilon} + \frac{1}{2} e^{-\lambda \varepsilon} -1 \right)} \right]
	\end{equation}
	Then, for $\lambda \in [b,a]$ we have:
	\begin{equation}
		\frac{p \left( \frac{\varepsilon}{2} e^{\lambda \varepsilon} - \frac{\varepsilon}{2} e^{-\lambda \varepsilon} \right) }{(1-p)(w_1 - w_2)e^{\lambda (w_1 - w_2)} }<\frac{p \left( \frac{1}{2} e^{\lambda \varepsilon} + \frac{1}{2} e^{-\lambda \varepsilon} -1 \right)}{p + (1-p) e^{\lambda (w_1 - w_2)} }
	\end{equation}
	which is
	\begin{equation}
		\frac{\frac{\partial}{\partial \lambda} \left[ p \left( \frac{1}{2} e^{\lambda \varepsilon} + \frac{1}{2} e^{-\lambda \varepsilon} -1 \right) \right]}{\frac{\partial}{\partial \lambda} \left[ p + (1-p) e^{\lambda (w_1 - w_2)} \right]} < \frac{p \left( \frac{1}{2} e^{\lambda \varepsilon} + \frac{1}{2} e^{-\lambda \varepsilon} -1 \right)}{p + (1-p) e^{\lambda (w_1 - w_2)} }
	\end{equation}
	Thus,  
	\begin{equation}
		\frac{p \left( \frac{1}{2} e^{\lambda \varepsilon} + \frac{1}{2} e^{-\lambda \varepsilon} -1 \right)}{p + (1-p) e^{\lambda (w_1 - w_2)} }
	\end{equation}
	is a decreasing function of $\lambda$ on $[b,a]$. Hence we must have:
	\begin{equation}
		\pi(b) > \pi(a)
	\end{equation}
	that is,
	\begin{equation}
		\pi_v > \pi_u
	\end{equation}
\end{homeworkSection}
\end{homeworkProblem}

\newpage

\begin{homeworkProblem}
Now the risk premium is defined as
\begin{equation}
	\frac{1}{2}u(w+\pi+\sigma) + \frac{1}{2}u(w+\pi-\sigma) = u(w)
\end{equation}
Suppose $\sigma$ is small enough, we can thus expect that $\pi$ is also small. It is easy to notice that $\sigma$ vanishes in the first order Taylor expansion of the LHS of the definition above. Thus, we expand to the second order:
\begin{equation}
	u(w) + u^{\prime}(w) \pi + \frac{1}{4}u^{\prime\prime}(w)\left[ (\pi+\sigma)^2 + (\pi-\sigma)^2 \right] + O((\pi\pm \sigma)^3) = u(w)
\end{equation}
This can be written as:
\begin{equation}
	u^{\prime}(w) \pi + \frac{1}{2}u^{\prime\prime}(w)\pi^2 + \frac{1}{2}u^{\prime\prime}(w)\sigma^2 + O((\pi \pm \sigma)^3) = 0
\end{equation}
Note that $\pi =O(\sigma ^2)$, thus $\pi^2$ term above can be dropped, and we have:
\begin{equation}
	u^{\prime}(w) \pi  + \frac{1}{2}u^{\prime\prime}(w)\sigma^2 \approx 0
\end{equation}
Thus we have
\begin{equation}
	\pi \approx -\frac{u^{\prime\prime}(w)}{2 u^{\prime}(w)}\sigma^2
\end{equation}


\end{homeworkProblem}

\newpage


%----------------------------------------------------------------------------------------

\end{document}
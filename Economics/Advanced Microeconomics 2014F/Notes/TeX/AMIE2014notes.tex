\documentclass{book}
  \usepackage{amsmath}
  \usepackage{amssymb}
  \usepackage{graphicx}
  \usepackage{float}
  \usepackage{setspace}
  \usepackage{stata}
  \usepackage{bm}
  \usepackage{amssymb}
  \usepackage{amsthm}
\topmargin=-1.2cm \oddsidemargin=0.1cm \evensidemargin=0.1cm
\textwidth=16 true cm \textheight=23 true cm

\font\euler=EUSM10 \font\eulers=EUSM7

\theoremstyle{plain}
\newtheorem{asm}{Assumption}[section]
\newtheorem{thm}{Theorem}[section]
\newtheorem{lem}[thm]{Lemma}
\newtheorem{prop}[thm]{Proposition}
\newtheorem*{cor}{Corollary}

\theoremstyle{definition}
\newtheorem{defn}{Definition}[section]
\newtheorem{conj}{Conjecture}[section]
\newtheorem{exmp}{Example}[section]

\theoremstyle{remark}
\newtheorem*{rem}{Remark}
\newtheorem*{note}{Note}

\begin{document}
\title{Advanced Microeconomics 2014 Fall\\Notes}
\author{{\normalsize SHENG Hao, 1401211818, via \LaTeX}}
\date{\today}

\maketitle
\def \Pr{{\rm Pr}}
\baselineskip 0.6cm
\tableofcontents
\chapter{Introduction}
\chapter{Preference and Utility}
\chapter{Indifference Curve and Demand Function}

\chapter{Production Problem}
\section{Production Technology}
\subsection{Production Function}
\begin{equation}
	Q = f(x_1,x_2,\dots,x_n)
\end{equation}
\begin{note}[1]
Cardinal property matters.
\end{note}
\begin{note}[2]
Monotonic technology:
\begin{equation}
	x^{'}_i\geq x_i \Rightarrow f(x^{'}_i, \overset{-}x_{-i})\geq f(x_i, \overset{-}x_{-i}), \quad \forall i = 1, 2, \dots, n
\end{equation}
\end{note}
\begin{note}[3]
Convex technology 
\end{note}

\begin{defn}[Input requirement set]
\begin{equation}
	\{ \bm{x}|f(\bm{x}) \geq y \}
\end{equation}
\end{defn}

\begin{defn}[Isoquant Set]
\begin{equation}
	\{ \bm{x}|f(\bm{x}) = y \}
\end{equation}
\end{defn}

\begin{defn}[Marginal rate of technological Substitution (MRTS)]
\begin{equation}
	\text{MRTS}_{x_1,x_2}=\frac{\frac{\partial f(x_1)}{\partial x_1}}{\frac{\partial f(x_2)}{\partial x_2}}
\end{equation}
\end{defn}

\begin{defn}[Elasticity of Substitution]
\begin{equation}
	E_{x_1,x_2}= \frac{\rm{d}(\frac{x_1}{x_2})}{{\rm d} \text{MRTS}_{x_1,x_2}}	
\end{equation}
\end{defn}

\subsection{Return to Scale}
\begin{defn}[Constant return to scale]
\begin{equation}
	\forall \alpha >1, \bm{x}\in \bm{X}, f(\alpha \bm{x})= \alpha f(\bm{x})
\end{equation}
\end{defn}

\begin{defn}[Increasing/Decreasing return to scale]
\begin{equation}
	\forall \alpha >1, \bm{x}\in \bm{X}, f(\alpha \bm{x})\gtrless \alpha f(\bm{x})
\end{equation}
\end{defn}

\begin{note}[1]
	The convexity/concavity of the function f(x) says nothing to the return of scale.
\end{note}

\begin{defn}[Elasticity of Scale(local concept)]
	Let $y(\alpha) = f(\alpha \bm{x})$ for any $\alpha > 0$, then the elasticity of scale at point $\bm{x}$ is  
	\begin{equation}
		e(y) = \frac{\frac{\rm{d} y(\alpha)}{y(\alpha)}}{\frac{\rm{d} \alpha}{ \alpha}}
	\end{equation}
\end{defn}

\begin{defn}[Homothetic Function]
Assume $h(\cdot)$ is homothetic of degree one, and $g(\cdot)$  is a increasing function, then $f(x)=g(h(x))$ is homothetic of degree one.
\end{defn}

\begin{note}[1]
	If $f(\bm{x}) = f(\bm{x^{'}})$, then $f(t\bm{x}) = f(t\bm{x^{'}})$ holds for any $t>0$.
\end{note}

\begin{note}[2]
The MRTS is the same for proportional change in $\bm{x}$.
\end{note}

\begin{note}[3]
Homethetic functions can be either constant increasing or decreasing return of scales.
\end{note}


\section{Profit Maximization}
\begin{asm}
	The objection of a firm is to maximize its profit.
\end{asm}
\begin{asm}
	All the firms are price taker of the price of outputs $p$ and the price of input factors $\bm{w}$
\end{asm}

We can consider the firm owner's decision in two ways, a one-step way is:
\begin{equation}
	\underset{\bm{x}}{Max}\quad p f(\bm{x}) - \bm{w}\cdot\bm{x}
\end{equation}
From this problem, we can derive the demand function of $\bm{x(p,\bm{w})}$, and profit function $\Pi(p, \bm{w})$

Or, we can derive a cost minimization responds for a given $f(x) = Q$:
\begin{align} \notag
	\underset{\bm{x}}{Min}\quad \bm{w}\cdot\bm{x}\\ \notag
	s.t. \quad f(\bm{x})=Q
\end{align}
So we get a conditional factor demand function $\bm{x}(Q, \bm{w})$ first, and corresponding cost function $C(Q, \bm{w})$. In the second step, we solve another optimization problem:
\begin{align} \notag
	\underset{Q}{Max}\quad pQ - C(Q, \bm{w})
\end{align}
This leads to the supply function $Q(p, \bm{w})$ and profit function $\Pi(p,\bm{w}) = pQ(p, \bm{w}) - C(Q, \bm{w})$, which is identical to the first way, as:
\begin{equation}
	\bm{x}(p,\bm{w}) = \bm{x}(Q(p,\bm{w}), \bm{w})
\end{equation}

There are three reasons that we try to solve this problem in two ways. First, the second method gives us the cost function which is crucial in analyzing the cost(MC, AC, FC, VC). Second, for certain technology that has increasing return of technology, there is no meaningful solution for the profit maximization problem but for the cost minimization one. Finally, the second method is not bounded by the price taker assumption.

\subsection{Solution of the problem}
For the profit maximization problem:
\begin{equation}
	\underset{\bm{x}}{Max}\quad p f(\bm{x}) - \bm{w}\cdot\bm{x}
\end{equation}
There is the First Order Condition({\bf F.O.C.}):
\begin{equation}
	p \cdot \nabla f(\bm{x})  = \bm{w}
\end{equation}
For a given profit level $\overset{-}{\Pi}$, since $\Pi = p f(\bm{x}) - \bm{w} \cdot \bm{x}$, 
\begin{equation}
	f(x_i,\overset{-}{x_{-i}}) = \frac{\overset{-}{\Pi}+}{p}
\end{equation} 
It's easy to check if there exist increasing return of scales, the optimal production level may be infinite. That's why we need the Second Order Condition. 
\subsection{Property of Profit Function}
\begin{prop}
$\Pi(p,\bm{w})$ is homogeneous of degree one.
\end{prop}
\begin{prop}
$\Pi(p,\bm{w})$ is convex in $(p,\bm{w})$, i.e. $\forall \lambda \in (0,1)$, let $p_0 = \lambda p_1 + (1-\lambda)p_2$, $\bm{w_0} = \lambda \bm{w_1} + (1-\lambda) \bm{w_2}$, we will have:
\begin{equation}
	\Pi(p_0,\bm{w_0})\leq \lambda \Pi(p_1,\bm{w_1}) + (1-\lambda)\Pi(p_2,\bm{w_2})
\end{equation}
Proof.\\
Denote $x_i$ the optimal choice for $(p_i,\bm{w_i}),\quad i = 0, 1, 2$ 
\begin{align} \notag
\Pi(p_0,\bm{w_0}) &= \lambda (p_1 f(\bm{x_0})-\bm{w_1}x_0) + (1-\lambda) (p_2 f(\bm{x_0})-\bm{w_2}x_0)\\ \notag
		& \leq \lambda (p_1 f(\bm{x_1}) -\bm{w_1}x_1) + (1-\lambda)(p_2 f(\bm{x_2})-\bm{w_2}x_2)\\ \notag
		& = \lambda \Pi(p_1,\bm{w_1}) + (1-\lambda)\Pi(p_2,\bm{w_2})
\end{align}
\end{prop}
\begin{note}[1]
If either wage or interest rate is cheap, since it's possible to substitute, it's better than no one is cheap.
\end{note}

\begin{prop}[Hotelling's Lemma]
Assume the $\Pi(p,\bm{w})$ is differentiable for any $p,\bm{w}$, we have
\begin{equation}
\frac{\partial \Pi(p,\bm{w})}{\partial p} = Q(p,\bm{w}),  \quad \frac{\partial \Pi(p,\bm{w})}{\partial w_i} = - x_i(p,\bm{w})
\end{equation}
Proof\footnote{More details need to be added.}.\\
Let $g(p) = \Pi(p,Q^{*}) - pQ^{*} - \bm{w^{*}}\bm{x^{*}}$. It's easy to check that $g(p)\geq 0$.\\
The equality condition is that $p=p^{*}$. Thus with the F.O.C, we have:
\begin{equation}
\frac{\partial \Pi(p,\bm{w})}{\partial p} = Q(p,\bm{w})
\end{equation}
The second equation can be proved in a similar way.
\end{prop}
\begin{note}[1]
The direct effect of price change is depicted by Hotelling's lemma, the indirect effect is the relative price change and the substitution.
\end{note}

\subsection{Factor Demand Function and Supply Function}
\begin{prop}
$x_i(p,\bm{w}), Q(p,\bm{w})$ is homogeneous of degree zero.
\end{prop}
\begin{prop}
$x_i(p,\bm{w})$ is non-increasing in $w_i$.\\
Proof.\\
 The F.O.C of the profit maximization problem is:
\begin{equation}
	p \nabla_{\bm{x}} f(\bm{x}) - \bm{w} = 0
\end{equation}
Differentiate its i-th equation with respect to $w_i$, we will have:
\begin{equation}
	p \frac{ \partial^2 f(x)}{\partial  x_i^2} \frac{\partial x_i(p,\bm{w})}{\partial w_i} = 1
\end{equation}
For the optimal $x$, we also have S.O.C that indicates,
\begin{equation}
	p \frac{\partial^2 f(x)}{\partial x^2}<0
\end{equation}
Thus the $\frac{\partial x_i(p,\bm{w})}{\partial w_i}$ is less than zero too.\\
Alternative Proof.\\
Using Hotelling's lemma, 
\begin{equation}
	\frac{\partial x_i}{\partial w_i} = \frac{\partial}{\partial w_i}(\frac{\partial \Pi(p, \bm{w})}{\partial w_i}) = \frac{\partial^2 \Pi(p, \bm{w})}{{\partial w_i}^2}>0
\end{equation}
, while the right hand side is greater than zero because of the convexity.

\end{prop}
\begin{note}[1]
There is no income effect for the non-existence of budget constraint.
\end{note}
\begin{note}[2]
Because it's due to the shape of production function, it's hard to decide the $p$'s effect on $\bm{x}$
\end{note}
\begin{prop}
$Q(p,\bm{w})$ is non-decreasing in $p$.
\end{prop}

\begin{prop}
The cross price effect is symmetric.\\
This is because, $\forall i,j$,
\begin{equation}
	\frac{\partial x_j}{\partial w_i} = \frac{\partial^2 \Pi(p, \bm{w})} {\partial w_i \partial w_j}=\frac{\partial^2 \Pi(p, \bm{w})} {\partial w_j \partial w_i} = \frac{\partial x_i}{\partial w_j}
\end{equation}
\end{prop}
\section{Cost Minimization\protect \footnotemark}\footnotetext{MWG 5C }
The cost minimization problem is described as following:
\begin{align}\notag
\underset{\bm{x}}{Min} \quad \bm{w}\cdot \bm{x} \\ \notag
s.t. f(\bm{x}) = Q
\end{align}
It's F.O.C:
\begin{equation}
	\text{MRTS}_{x_i,x_j}=\frac{\frac{\partial f(x_i)}{\partial x_i}}{\frac{\partial f(x_j)}{\partial x_j}} = \frac{w_i}{w_j}
\end{equation}
\subsection{Conditional factor demand function and cost function}
\begin{prop}
$C(Q,\bm{w})$ is homogeneous of degree 1 in $\bm{w}$.
\end{prop}

\begin{prop}
$C(Q,\bm{w})$ is concave in $\bm{w}$, i.e. let $\bm{w_0} = \lambda \bm{w_1} + (1-\lambda)\bm{w_2}, \forall \lambda \in (0,1)$, 
\begin{equation}
	C(Q,\bm{w_0}) \geq \lambda C(Q,\bm{w_1}) + (1-\lambda)C(Q,\bm{w_2})
\end{equation}
\end{prop}

\begin{prop}[Shephard's Lemma]
Suppose the cost function is diffentiable at $(Q,\bm{w})$, for $w_i>0$, we have,
\begin{equation}
	\frac{\partial C(Q,\bm{w})}{\partial w_i} = x_i(Q,\bm{w})\geq 0
\end{equation}
\end{prop}
and its three conjecture,
\begin{conj}
$C(Q,\bm{w})$ is non-decreasing in $w_i$.
\end{conj}
\begin{conj}
\begin{equation}
	\frac{\partial x_i(Q,\bm{w})}{\partial w_i} = \frac{\partial^2 C(Q,\bm{w})}{{\partial w_i}^2}\leq 0 (Concaveness)
\end{equation}
\end{conj}
\begin{conj}
Cross price effect is symmetric: 
\begin{equation}
\frac{\partial x_j}{\partial w_i} = \frac{\partial^2 C(Q, \bm{w})} {\partial {w_i} \partial {w_j}}=\frac{\partial^2 C(Q, \bm{w})} {\partial w_j \partial w_i} = \frac{\partial {x_i}}{\partial {w_j}}
\end{equation}
\end{conj}

\subsection{Cost in the long run and short run}
\begin{defn}[Average Cost]
\begin{equation}
	\text{AC} = \frac{C(Q,\bm{w})}{Q}
\end{equation}
\end{defn}
\begin{defn}[Marginal Cost]
\begin{equation}
	\text{MC} = \frac{\partial C(Q,\bm{w})}{\partial Q}
\end{equation}
\end{defn}

\begin{prop}
AC keeps rising when MC$>$AC and vice versa.
\begin{equation}
	\frac{\partial AC}{\partial Q} = \frac{MC-AC}{Q}
\end{equation}
\end{prop}

\begin{defn}[Variable Cost]
\begin{equation}
	VC  = \bm{w_v}\cdot \bm{x_v}
\end{equation}
\end{defn}
\begin{defn}[Fix Cost]
\begin{equation}
	FC  = \bm{w_f}\cdot \bm{x_f}
\end{equation}
\end{defn}

\begin{defn}[Short-run and Long-run]
In the short-run, at least one input is fixed; in the long-run, all inputs can be adjusted: there is no fixed cost.
\end{defn}

\begin{defn}[Short-run Total Cost]
\begin{equation}
	\text{STC} = C(Q,\bm{w}, \bm{x_f}) = \bm{w_v}\bm{x_v(Q,\bm{w},\bm{x_f})} + \bm{w_f}\bm{x_f}
\end{equation}
\end{defn}

\begin{defn}[Short-run Average Cost]
\begin{equation}
	\text{SATC} = \frac{\text{STC}}{Q}
\end{equation}
\end{defn}

\begin{defn}[Short-run Average Variable Cost]
\begin{equation}
	\text{SAVC} = \frac{\bm{w_v}\bm{x_v(Q,\bm{w},\bm{x_f})}}{Q}
\end{equation}
\end{defn}


\begin{defn}[Short-run Average Fixed Cost]
\begin{equation}
	\text{SAFC} = \frac{\bm{w_f}\bm{x_f}}{Q}
\end{equation}
\end{defn}

\begin{defn}[Short-run Marginal Cost]
\begin{equation}
	\text{SMC} = \frac{\partial \text{STC}}{\partial Q}
\end{equation}
\end{defn}

\begin{defn}[Long-run Total Cost]
\begin{equation}
	\text{LTC} = C(Q,\bm{w}) = \bm{w_v}\bm{x_v}(Q,\bm{w},\bm{x_f(Q,\bm{W}))} + \bm{w_f}\bm{x_f}(Q,\bm{w})
\end{equation}
\end{defn}

\begin{prop}
Denote $\bm{x^{*}}$ as the optimal long-run choice for $(Q^{*},\bm{w})$, 
\begin{equation}
	C(Q,\bm{w},\bm{x^{*}_f}) \geq C(Q,\bm{w},\bm{x_f}(Q,\bm{w}))
\end{equation}
Equality is satisfied when $Q = Q^{*}$
\end{prop}

\begin{defn}[Accounting Profit and Economic Profit]\quad\\
Accounting profit = Revenue - Accounting cost\\
Economic Profit = Revenue - Accounting cost - Opportunity cost
\end{defn}

\begin{defn}[Sunk Cost]
\end{defn}

\section{Supply Curve (Optimal Q)}
This problem could be described as
\begin{equation}
	\underset{Q}{Max} \quad PQ - C(Q,\bm{w})
\end{equation}
Its F.O.C:
\begin{equation}
	P = \frac{\partial C(Q,\bm{w})}{\partial Q} = MC
\end{equation}

\chapter{Uncertainty}
\begin{defn}[Risk, Uncertainty and Ambiguity]
Risk is objective probability distribution, while uncertainty is subjective probability distribution. We say ambiguity if one has no probability distribution.
\end{defn}

\section{Lottery Representation}
\begin{defn}[Lottery]
Consider a Lottery $X$:
\begin{equation}
	X = (x_1, p_1, x_2, p_2, \cdots, x_n, p_n)
\end{equation}
, where $p_i \in [0,1]$ and $\sum^n_{i=1}p_i=1$.\\
$x_i$ is called price, it could be a consumption bundle, income or further lottery.
\end{defn}
e.g. $P_A = (x_1,\frac{1}{2}, x_2, \frac{1}{2})$, $P_B =(x_3,1)$. We can rewrite them as:
\begin{align} \notag
P_A &= (x_1,\frac{1}{2}, x_2, \frac{1}{2}, x_3, 0) \\ \notag
P_B &= (x_1,0, x_2, 0, x_3, 1)
\end{align}
What we care is the trinary pair $(\frac{1}{2},\frac{1}{2},0)$ and $(0,0,1)$.
\begin{defn}[Compound Lottery]
Suppose $P_A = (x_1,\frac{1}{2}, x_2, \frac{1}{2})$, where $x_2$ is another lottery defined as $x_2=( y_1, \frac{2},{3}, y_2,\frac{1}{3})$. The reduced form of $P_A$ would be:
	\begin{equation}
		P_A=(x_1, \frac{1}{2}, y_1, \frac{1},{3}, y_2,\frac{1}{6})
	\end{equation}
\end{defn}

\begin{defn}{Preference Function}
When the space of lotteries satisfy the following two, say:
Completeness: For any $P_A, P_B$, there must be $P_A< P_B$, $P_A>P_B$ or $P_A= P_B$\\
Transitive: $P_A \geq P_B, P_B \geq P_C \Rightarrow P_A \geq P_C$
There is a preference function $V(x_1, p_1, \cdots,x_n, p_n)$ just as the utility function.
\end{defn}
\subsection{First Order Stochastic Dominance (about mean)}
\begin{asm}
Let each $x_i$ be defined over the wealth space where we can rank them. Lottery can be represented by distribution PDF $f(\cdot)$ and CDF $F(\cdot)$.
\end{asm}
\begin{defn}[First Order Stochastic Dominance]
$f(\cdot)$ is said first-order stochastically dominated by  $f^{*}(\cdot)$ if $f(\cdot)$ can be obtained from $F^{*}(\cdot)$ by rightward shifts of probability mass, that is , shift of probability mass from lower outcome values to higher outcome values.
\end{defn}
\begin{note}[1]
This property has nothing to do with preference functions.
\end{note}
\subsection{Second-order Stochastic Dominance (about variance)}
\begin{defn}[Mean-preserving spread]
Mean-preserving spread is such a way that moves probability mass from the center of a probability distribution t its tails in a away that preserves the expected value of the distribution.
\end{defn}

\begin{defn}[Second-order Stochastic Dominance]
We say $F(\cdot)$ second-order stochastically dominates $F^{*}(\cdot)$ if $F^{*}(\cdot)$  is a mean preserving spread from $F(\cdot)$
\end{defn}
\begin{note}[1]
We can say $F^{*}(\cdot)$ is risker than $F(\cdot)$, keeping expected value the same.
\end{note}
\begin{note}[2]
Mean presenting spread can be represented as X~random variable with distribution $F(\cdot)$. Let $X^{*} = X+ e$, where$E(e|x)=0, Var(e|X)=0$. Then $F^{*}(x^{*})$ is a mean presenting spread of $F(\cdot)$
\end{note}
\begin{note}[3]
Any individual with risk averse preference would prefer $F(\cdot)$ to $F^{*}(\cdot)$
\end{note}

\section{Expected Utility Model}
\subsection{Independent Axiom}
If $L>L^{*}$, then for any $\alpha \in (0,1)$ and any $L^{*}$:
\begin{equation}
	\alpha L + (1-\alpha) L^{**} > \alpha L^{*} + (1-\alpha) L^{**}
\end{equation}

\begin{note}[1]
This property implies linearity in probability.
\end{note}
\section{Subjective Probability Theory}
\section{Lab and Field Evidence}
\end{document}


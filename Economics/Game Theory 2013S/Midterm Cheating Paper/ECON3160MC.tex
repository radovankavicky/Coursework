\documentclass{article}
  \usepackage{amsmath}
  \usepackage{amssymb}
  \usepackage{graphicx}
  \usepackage{float}
  \usepackage{setspace}
  \usepackage{verbatim}
\topmargin=-1.2cm \oddsidemargin=0.1cm \evensidemargin=0.1cm
\textwidth=16 true cm \textheight=23 true cm

\font\euler=EUSM10 \font\eulers=EUSM7

\begin{document}
\title{ECON 3160 Game Theory \\Midterm Cheating Paper}
\author{{\normalsize Leonard Sheng(SHENG, Hao), 1155035947, via \LaTeX}}
\date{\today}

\maketitle

\def \Pr{{\rm Pr}}

\baselineskip 0.6cm
\begin{description}
  \item[Theorem 2.5.3](Characterization of rational choices)\\
  The {\bf rational choices} in a static game are exactly those choices that are neither strictly dominated by some other choice nor strictly dominated by a randomized choice.
  \item[Definition 3.4.2] (k-fold belief in rationality)\\
  1)Type $t_i$ expresses {\bf 1-fold belief} in rationality if $t_i$ believes in the opponents' rationality.\\
  2)Type $t_i$ expresses {\bf 2-fold belief} in rationality if $t_i$ only assigns positive probability to the opponents' types that express 1-fold belief in rationality.\\
  And so on.
  \item[Definition of *k-th order belief](In slides)\\
  1)1st order belief is your belief about your opponents' choices.\\
  2)2nd order belief is your belief about your opponents' 1st order beliefs.\\
  And so on.
  \item[Theorem 3.6.1] (Common belief in rationality is always possible)\\
  Consider a static game with finitely many choices for every player. Then, we can construct an epistemic model in which:\\
  1)every type expresses common belief in rationality, and\\
  2)every type assigns, for every opponent, probability 1 to one specific choice and one specific type for that opponent.
  \item[Theorem 3.7.2] (The algorithm works)\\
  Choices that survive: \\
  1 round of elimination are those consistent with rationality;\\
  2 rounds of elimination are those consistent with (rationality + 1-fold belief in rationality);\\
  and so on...\\
  k + 1 rounds of elimination are those consistent with (rationality + k-fold belief in rationality);\\
  ...\\
  K rounds of elimination are those consistent with (rationality + common belief in rationality).
  \item[Theorem 3.8.1] (Order and speed of elimination does not matter)\\
  The set of choices that survive the algorithm of iterated elimination of strictly dominated choices does not change if we change the order and speed of elimination.
  \item[Definition 4.1.2] (Simple belief hierarchy)\\
  Consider an epistemic model with sets of types $T_1,...,T_n$. A type $t_i \in T_i$ for player$i$ is said to have a {\bf simple belief hierarchy} if $t_i$'s belief hierarchy is generated by some combination ($\sigma_1,...,\sigma_n$) of beliefs about the players' choices.
  \item[Definition 4.2.1] (Nash equilibrium)\\
  Consider a game with $n$ players. Let ($\sigma_1,...,\sigma_n$) be a combination of beliefs where $\sigma_i$ is a probabilistic belief about player$i$'s choice. The combination of beliefs is a {\bf Nash equilibrium} if for every player$j$, the belief $\sigma_j$ only assigns positive probability to choices $c_j$ that are optimal for player$j$ under the belief $\sigma_{-j}$ about the opponents' choices.
  \item[Theorem 4.2.2] (Simple belief hierarchies versus Nash equilibrium)\\
  A simple belief hierarchy expresses common belief in rationality if and only if the combination of beliefs ($\sigma_1,...,\sigma_n$) is a Nash equilibrium.
  \item[Theorem 4.2.3] (Nash equilibrium always exists)\\
  For every game with finitely many choices there is at least one Nash equilibrium, and hence there is at least one Nash choice for each player.
  \item[Lemma 4.4.2] (Characterization of types that believe that opponents hold correct beliefs)\\
  A type $t_i$ for player$i$ believes that his opponents hold correct beliefs if and only if type $t_i$ believes that every opponent believes that i's type is $t_i$.
  \item[Definition of *Naked] (In slides)\\
  A simple person feels naked when he feels all his opponents believe that he has the belief hierarchy which is exactly what he has.
  \item[Definition of *Agreeable] (In slides)\\
  A simple person feels agreeable when he first-order belief about player$j$ agree with (what he thinks are) everyone else's first-order belief about player$j$'s choice.
  \item[Theorem 4.4.5](roughly)\\
  If type $t_i$ a)feels naked; b)believes his opponents feel naked; c)feels agreeable; d)believes his opponents feel agreeable; e)does not believe in correlation; f)does not believe his opponents believe in correlation, then this type has simple belief hierarchy.
\end{description}
\end{document}

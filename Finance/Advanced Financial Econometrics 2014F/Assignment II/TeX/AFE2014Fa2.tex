\documentclass{article}
  \usepackage{amsmath}
  \usepackage{amssymb}
  \usepackage{graphicx}
  \usepackage{float}
  \usepackage{setspace}
  \usepackage{stata}
  \usepackage{natbib}
  \usepackage{indentfirst}
  \usepackage[colorlinks, linkcolor=blue, anchorcolor=blue, citecolor=green]{hyperref}%For hyper-link reference
\setlength{\parindent}{2em}

\topmargin=-1.2cm \oddsidemargin=0.1cm \evensidemargin=0.1cm
\textwidth=16 true cm \textheight=23 true cm

\font\euler=EUSM10 \font\eulers=EUSM7

\begin{document}
\title{Financial Econometrics\\Assignment $2^{\text{nd}}$}
\author{{\normalsize SHENG Hao, 1401211818, via \LaTeX}}
\date{\today}

\maketitle

\def \Pr{{\rm Pr}}
\baselineskip 0.6cm

This assignment is for the graduate course Financial Econometrics at GSM(Guanghua School of Management, Peking Univ.). In this assignment, we are asked to explain the capital structure of listed Chinese firms(SSE\&SZSE) by firm-level variables, and try to solve the endogenous problem. The answer to the four questions are in the sections: {\bf \nameref{sec:model}, \nameref{sec:test}, \nameref{sec:endogenous}, \nameref{sec:IV}}, respectively.


\section{Introduction}
\subsection{Data Description and Clean-up}
Our data covers all the listed stocks in SSE(Shanghai Stock Exchange) and SZSE(Shenzhen Stock Exchange)in from Jan.1$^{\text{st}}$ 2001 to Dec.31$^{\text{th}}$ 2013. Since banking industry may have a completely different pattern of capital structure(Wei 2013), we drop all the observations in that industry(up to 35 individual stocks, less than 1\% of all). We also drop those of insolvency, that is to say, those with asset less than liability. 
%drop those IPO oversea
We also made a 1\% level {\it Winsor transformation}(Barnett and Lewis, 1994) to variable we care, in order to rule out the outliers' effect.

\subsection{Explanatory Variables}
Literature have tested enough factors that might have a casual relation with the capital leverage. In (Frank\&Goyal, 2004), 37 explanatory variables are included covering firm-level(size, profitability, governance, tax, etc.), industry-level and market-level(institution and marketization) variance.

Since our aim is not exhaust the such possible factors, apart variables of interest(that is, {\it size, age, age$^2$, ROE}) , we only pick up several control variables like the ownership concentration, liquidity ratio, PB ratio, etc. We hope the coefficients of the variables we care and their significance change little when more are controlled. (This is another way to say controlling additional variable only serves as a placebo test, and might save us of the criticism of missing important variables.)

In part \ref{sec:IV}, we will revisit the endogenous problem, and try to solve it with instrument variables.

Here is a general summarize of the variables:


\begin{table}[htbp]\centering
\caption{\label{tab:latabstat1} 
\text{Summarize of Variables} }\begin{tabular} {@{} l r r r r r @{}} \\ \hline \hline
\textbf{variable } & \textbf{mean} & \textbf{sd} & \textbf{p50} & \textbf{min} & \textbf{max} \\
\hline 
           DERatio  & 0.47 & 0.21 & 0.48 & 0.02 & 1.00 \\
           CorpAge  & 11.95 & 5.12 & 11.62 & 0.15 & 31.61 \\
          CorpAge2  & 168.99 & 134.14 & 135.07 & 0.02 & 998.91 \\
              Size  & 21.59 & 1.32 & 21.40 & 18.28 & 30.57 \\
               ROE  & 0.07 & 0.17 & 0.08 & -1.82 & 3.12 \\
    LiquidityRatio  & 2.23 & 3.18 & 1.37 & 0.05 & 56.60 \\
             HHI10  & 0.19 & 0.13 & 0.16 & 0.01 & 0.68 \\
           PBRatio  & 0.74 & 0.52 & 0.73 & 0.05 & 38.55 \\
              SGAe  & 0.20 & 0.74 & 0.14 & -0.77 & 83.43 \\
         CEOGender  & 0.85 & 0.10 & 0.87 & 0.33 & 1.00 \\
            CEOAge  & 47.66 & 3.45 & 47.77 & 35.36 & 59.57 \\
         CEOSalary  & 167382.07 & 201519.47 & 116855.47 & 0.00 & 7.17e+06 \\
          CEOShare  & 612985.40 & 2.16e+06 & 811.73 & 0.00 & 4.14e+07 \\
\hline \hline
\multicolumn{6}{@{}l}{\footnotesize{\emph{Source:} }}
\end{tabular}
\end{table}





Before estimation, we can make a educated guess about the direction of our explain variables' effects. 
\begin{description}
\item [Performance]: Given that firms with a better performance raise money much more easier from the stock market, we suppose the coefficient of {\it ROE} would be positive. 
\item [Size]: Banks may prefer lending to large firms for the 	
\item []
\item [Capital structure]:
\item [CEO Characters]:
\item [Concentration of shares]:
\item [Liquidity]
\end{description}
\subsection{Control Variables}
\newpage
\section{Specification and Results} \label{sec:model}
\begin{table}[htbp]\centering
\def\sym#1{\ifmmode^{#1}\else\(^{#1}\)\fi}
\caption{Results of OLS, FE and RE \label{tab:regression}}
\begin{tabular}{l*{6}{c}}
\hline\hline
            &\multicolumn{1}{c}{(1)}&\multicolumn{1}{c}{(2)}&\multicolumn{1}{c}{(3)}&\multicolumn{1}{c}{(4)}&\multicolumn{1}{c}{(5)}&\multicolumn{1}{c}{(6)}\\
            &\multicolumn{1}{c}{OLS}&\multicolumn{1}{c}{OLS$^{1}$}&\multicolumn{1}{c}{FE}&\multicolumn{1}{c}{FE$^{1}$}&\multicolumn{1}{c}{RE}&\multicolumn{1}{c}{RE$^{1}$}\\
\hline
CorpAge     &       0.019***&       0.012***&       0.016***&       0.011***&       0.025***&       0.018***\\
            &     (0.001)   &     (0.001)   &     (0.001)   &     (0.001)   &     (0.001)   &     (0.001)   \\
[1em]
CorpAge2    &      -0.000***&      -0.000***&      -0.001***&      -0.000***&      -0.000***&      -0.000***\\
            &     (0.000)   &     (0.000)   &     (0.000)   &     (0.000)   &     (0.000)   &     (0.000)   \\
[1em]
Size        &       0.057***&       0.051***&       0.060***&       0.069***&       0.065***&       0.068***\\
            &     (0.001)   &     (0.002)   &     (0.002)   &     (0.003)   &     (0.001)   &     (0.002)   \\
[1em]
ROE         &      -0.035***&      -0.037***&      -0.018***&      -0.018***&      -0.020***&      -0.020***\\
            &     (0.002)   &     (0.004)   &     (0.001)   &     (0.002)   &     (0.001)   &     (0.002)   \\
[1em]
LiquidityRatio&               &      -0.016***&               &      -0.012***&               &      -0.012***\\
            &               &     (0.000)   &               &     (0.000)   &               &     (0.000)   \\
[1em]
HHI10       &               &      -0.128***&               &      -0.112***&               &      -0.133***\\
            &               &     (0.015)   &               &     (0.019)   &               &     (0.017)   \\
[1em]
PBRatio     &               &       0.006***&               &       0.005***&               &       0.005***\\
            &               &     (0.002)   &               &     (0.001)   &               &     (0.001)   \\
[1em]
SGAe        &               &       0.000***&               &       0.000***&               &       0.000***\\
            &               &     (0.000)   &               &     (0.000)   &               &     (0.000)   \\
[1em]
CEOGender   &               &       0.086***&               &       0.067***&               &       0.063***\\
            &               &     (0.018)   &               &     (0.020)   &               &     (0.018)   \\
[1em]
CEOAge      &               &      -0.002***&               &      -0.002***&               &      -0.002***\\
            &               &     (0.001)   &               &     (0.001)   &               &     (0.001)   \\
[1em]
CEOSalary   &               &      -0.000***&               &      -0.000***&               &      -0.000***\\
            &               &     (0.000)   &               &     (0.000)   &               &     (0.000)   \\
[1em]
CEOShare    &               &      -0.000***&               &      -0.000***&               &      -0.000***\\
            &               &     (0.000)   &               &     (0.000)   &               &     (0.000)   \\
\hline
\(N\)       &       21496   &        9946   &       21496   &        9946   &       21496   &        9946   \\
adj. \(R^{2}\)&       0.278   &       0.392   &       0.050   &       0.095   &               &               \\
\hline\hline
\multicolumn{7}{l}{\footnotesize Standard errors in parentheses}\\
\multicolumn{7}{l}{\footnotesize * \(p<0.1\), ** \(p<0.05\), *** \(p<0.01\)}\\
\end{tabular}
\end{table}

At this point, we cannot rush to the conclusion which model did a better job in estimation. The following part conduct a further hypothesis test in specification.

\newpage
\section{Hausman Test}\label{sec:test}
\begin{center}
	\begin{stlog}
	. hausman fe re, sigmamore all
{\smallskip}
Note: the rank of the differenced variance matrix (19) does not equal the
        number of coefficients being tested (22); be sure this is what you
        expect, or there may be problems computing the test.  Examine the
        output of your estimators for anything unexpected and possibly consider
        scaling your variables so that the coefficients are on a similar scale.
{\smallskip}
                 \HLI{4} Coefficients \HLI{4}
             {\VBAR}      (b)          (B)            (b-B)     sqrt(diag(V_b-V_B))
             {\VBAR}       fe           re         Difference          S.E.
\HLI{13}{\PLUS}\HLI{64}
         ROE {\VBAR}    .1633808     .1151406        .0482401        .0077012
     CorpAge {\VBAR}    .0455443     .0161776        .0293667        .0110587
    CorpAge2 {\VBAR}   -.0002991    -.0002536       -.0000454               .
        Size {\VBAR}    .0750829     .0716512        .0034317        .0003929
LiquidityR{\tytilde}o {\VBAR}   -.0168089     -.017897        .0010881               .
       HHI10 {\VBAR}   -.1760706    -.1721294       -.0039412        .0076773
     PBRatio {\VBAR}    .0075511     .0069848        .0005663               .
        SGAe {\VBAR}    .0825913     .0545654        .0280259        .0077674
   CEOGender {\VBAR}    .0724868     .0656559        .0068309               .
      CEOAge {\VBAR}   -.0020626    -.0020026         -.00006               .
   CEOSalary {\VBAR}   -1.23e-07    -1.16e-07       -6.65e-09               .
    CEOShare {\VBAR}   -4.06e-09    -4.88e-09        8.23e-10        1.90e-10
 2004bn.year {\VBAR}   -.0196582     .0082732       -.0279314         .010887
   2005.year {\VBAR}   -.0321722     .0228997       -.0550719        .0224422
   2006.year {\VBAR}   -.0726744     .0114534       -.0841278        .0333425
   2007.year {\VBAR}   -.1294581    -.0149229       -.1145351        .0443114
   2008.year {\VBAR}   -.1597309    -.0190632       -.1406676         .055951
   2009.year {\VBAR}   -.1989022    -.0296251       -.1692771        .0670902
   2010.year {\VBAR}    -.233997    -.0350244       -.1989726        .0780411
   2011.year {\VBAR}   -.2807397    -.0552167        -.225523        .0894404
   2012.year {\VBAR}   -.3141874    -.0633026       -.2508848        .1009273
   2013.year {\VBAR}   -.3523648    -.0736676       -.2786972        .1122267
\HLI{13}{\BOTT}\HLI{64}
                         b = consistent under Ho and Ha; obtained from xtivreg
          B = inconsistent under Ha, efficient under Ho; obtained from xtivreg
{\smallskip}
    Test:  Ho:  difference in coefficients not systematic
{\smallskip}
                 chi2(19) = (b-B)'[(V_b-V_B){\caret}(-1)](b-B)
                          =   -48.67    chi2<0 ==> model fitted on these
                                        data fails to meet the asymptotic
                                        assumptions of the Hausman test;
                                        see suest for a generalized test
\nullskip
	\end{stlog}
\end{center}
As is shown in the result above, the statistics given by the {\bf Hausman Test} corresponds to a p-value of less than 0.01 percent. Thus, we can refuse the $H_0$, which is to say, we can rule out the potential existence of {\it Random Effect} upon 0.01 percent level and there is no need to do the further {\bf Breusch-Pagan Test}. 
Therefore, We stop here and would like to report the result of {\bf Fixed Effect Model}:
\begin{table}[htbp]\centering
\def\sym#1{\ifmmode^{#1}\else\(^{#1}\)\fi}
\caption{Result of Fixed Effect Model \label{tab:regressionFE}}
\begin{tabular}{l*{3}{c}}
\hline\hline
            &\multicolumn{1}{c}{(1)}&\multicolumn{1}{c}{(2)}&\multicolumn{1}{c}{(3)}\\
            &\multicolumn{1}{c}{FE}&\multicolumn{1}{c}{FE}&\multicolumn{1}{c}{IV-FE}\\
\hline
CorpAge     &      0.0321***&      0.0492***&      0.0455***\\
            &    (0.0065)   &    (0.0122)   &    (0.0135)   \\
[1em]
CorpAge2    &     -0.0004***&     -0.0002***&     -0.0003***\\
            &    (0.0000)   &    (0.0000)   &    (0.0001)   \\
[1em]
Size        &      0.0672***&      0.0741***&      0.0751***\\
            &    (0.0016)   &    (0.0025)   &    (0.0028)   \\
[1em]
ROE         &     -0.1411***&     -0.1171***&      0.1634** \\
            &    (0.0049)   &    (0.0067)   &    (0.0678)   \\
[1em]
LiquidityRatio&               &     -0.0169***&     -0.0168***\\
            &               &    (0.0005)   &    (0.0006)   \\
[1em]
HHI10       &               &     -0.0666***&     -0.1761***\\
            &               &    (0.0190)   &    (0.0336)   \\
[1em]
PBRatio     &               &      0.0059***&      0.0076***\\
            &               &    (0.0021)   &    (0.0023)   \\
[1em]
SGAe        &               &      0.0308***&      0.0826***\\
            &               &    (0.0051)   &    (0.0136)   \\
[1em]
CEOGender   &               &      0.0702***&      0.0725***\\
            &               &    (0.0188)   &    (0.0207)   \\
[1em]
CEOAge      &               &     -0.0018***&     -0.0021***\\
            &               &    (0.0007)   &    (0.0007)   \\
[1em]
CEOSalary   &               &     -0.0000***&     -0.0000***\\
            &               &    (0.0000)   &    (0.0000)   \\
[1em]
CEOShare    &               &     -0.0000***&     -0.0000***\\
            &               &    (0.0000)   &    (0.0000)   \\
\hline
\(N\)       &       21496   &        9946   &        9936   \\
adj. \(R^{2}\)&       0.088   &       0.160   &               \\
\hline\hline
\multicolumn{4}{l}{\footnotesize Standard errors in parentheses}\\
\multicolumn{4}{l}{\footnotesize * \(p<0.1\), ** \(p<0.05\), *** \(p<0.01\)}\\
\end{tabular}
\end{table}


\newpage
\section{Endogenous Problem}\label{sec:endogenous}
It's arguable that all the main explaining variables we care may suffer from endogenous problem through the channel of uncontrolled CEO preference. But most 

\section{IV Regression}\label{sec:IV}

\begin{appendix}\label{sec:appendix}
\nocite{*}
\bibliography{Library}
\bibliographystyle{jpe}
\end{appendix}
\end{document}

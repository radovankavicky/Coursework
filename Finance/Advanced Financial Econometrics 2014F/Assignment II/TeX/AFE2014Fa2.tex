\documentclass{article}
  \usepackage{amsmath}
  \usepackage{amssymb}
  \usepackage{graphicx}
  \usepackage{float}
  \usepackage{setspace}
  \usepackage{stata}
\topmargin=-1.2cm \oddsidemargin=0.1cm \evensidemargin=0.1cm
\textwidth=16 true cm \textheight=23 true cm

\font\euler=EUSM10 \font\eulers=EUSM7

\begin{document}
\title{Financial Econometrics\\Assignment $2^{\text{nd}}$}
\author{{\normalsize SHENG Hao, 1401211818, via \LaTeX}}
\date{\today}

\maketitle

\def \Pr{{\rm Pr}}
\baselineskip 0.6cm

This assignment is for the graduate course Financial Econometrics at GSM(Guanghua School of Management, Peking Univ.). In this assignment, we are asked to explain the capital structure of listed Chinese firms(SSE\&SZSE) by firm-level variable, and try to solve the endogenous problem. 
\section{Introduction}
\subsection{Data Description}
Our data covers all the listed stocks in SSE(Shanghai Stock Exchange) and SZSE(Shenzhen Stock Exchange)in from Jan.1$^{\text{st}}$ 2001 to Dec.31$^{\text{th}}$ 2013. Since banking industry may have a completely different pattern of capital structure(Wei 2013), we drop all the observations in that industry(up to 35 individual stocks, less than 1\% of all). We also drop those of insolvency, that is to say, those with asset less than liability. 
\subsection{Explain Variables}
\subsection{Control Variables}
\newpage
\section{OLS, Fix Effect Model or Random Effect Model}
\begin{table}[htbp]\centering
\def\sym#1{\ifmmode^{#1}\else\(^{#1}\)\fi}
\caption{Results of OLS, FE and RE \label{tab:regression}}
\begin{tabular}{l*{6}{c}}
\hline\hline
            &\multicolumn{1}{c}{(1)}&\multicolumn{1}{c}{(2)}&\multicolumn{1}{c}{(3)}&\multicolumn{1}{c}{(4)}&\multicolumn{1}{c}{(5)}&\multicolumn{1}{c}{(6)}\\
            &\multicolumn{1}{c}{OLS}&\multicolumn{1}{c}{OLS$^{1}$}&\multicolumn{1}{c}{FE}&\multicolumn{1}{c}{FE$^{1}$}&\multicolumn{1}{c}{RE}&\multicolumn{1}{c}{RE$^{1}$}\\
\hline
CorpAge     &       0.019***&       0.012***&       0.016***&       0.011***&       0.025***&       0.018***\\
            &     (0.001)   &     (0.001)   &     (0.001)   &     (0.001)   &     (0.001)   &     (0.001)   \\
[1em]
CorpAge2    &      -0.000***&      -0.000***&      -0.001***&      -0.000***&      -0.000***&      -0.000***\\
            &     (0.000)   &     (0.000)   &     (0.000)   &     (0.000)   &     (0.000)   &     (0.000)   \\
[1em]
Size        &       0.057***&       0.051***&       0.060***&       0.069***&       0.065***&       0.068***\\
            &     (0.001)   &     (0.002)   &     (0.002)   &     (0.003)   &     (0.001)   &     (0.002)   \\
[1em]
ROE         &      -0.035***&      -0.037***&      -0.018***&      -0.018***&      -0.020***&      -0.020***\\
            &     (0.002)   &     (0.004)   &     (0.001)   &     (0.002)   &     (0.001)   &     (0.002)   \\
[1em]
LiquidityRatio&               &      -0.016***&               &      -0.012***&               &      -0.012***\\
            &               &     (0.000)   &               &     (0.000)   &               &     (0.000)   \\
[1em]
HHI10       &               &      -0.128***&               &      -0.112***&               &      -0.133***\\
            &               &     (0.015)   &               &     (0.019)   &               &     (0.017)   \\
[1em]
PBRatio     &               &       0.006***&               &       0.005***&               &       0.005***\\
            &               &     (0.002)   &               &     (0.001)   &               &     (0.001)   \\
[1em]
SGAe        &               &       0.000***&               &       0.000***&               &       0.000***\\
            &               &     (0.000)   &               &     (0.000)   &               &     (0.000)   \\
[1em]
CEOGender   &               &       0.086***&               &       0.067***&               &       0.063***\\
            &               &     (0.018)   &               &     (0.020)   &               &     (0.018)   \\
[1em]
CEOAge      &               &      -0.002***&               &      -0.002***&               &      -0.002***\\
            &               &     (0.001)   &               &     (0.001)   &               &     (0.001)   \\
[1em]
CEOSalary   &               &      -0.000***&               &      -0.000***&               &      -0.000***\\
            &               &     (0.000)   &               &     (0.000)   &               &     (0.000)   \\
[1em]
CEOShare    &               &      -0.000***&               &      -0.000***&               &      -0.000***\\
            &               &     (0.000)   &               &     (0.000)   &               &     (0.000)   \\
\hline
\(N\)       &       21496   &        9946   &       21496   &        9946   &       21496   &        9946   \\
adj. \(R^{2}\)&       0.278   &       0.392   &       0.050   &       0.095   &               &               \\
\hline\hline
\multicolumn{7}{l}{\footnotesize Standard errors in parentheses}\\
\multicolumn{7}{l}{\footnotesize * \(p<0.1\), ** \(p<0.05\), *** \(p<0.01\)}\\
\end{tabular}
\end{table}

At this point, we cannot rush to the conclusion which model did a better job in estimation. The following part conduct a further hypothesis test in specification.

\newpage
\section{Hausman Test}
\begin{center}
	\begin{stlog}
	. hausman fe re, sigmamore all
{\smallskip}
Note: the rank of the differenced variance matrix (19) does not equal the
        number of coefficients being tested (22); be sure this is what you
        expect, or there may be problems computing the test.  Examine the
        output of your estimators for anything unexpected and possibly consider
        scaling your variables so that the coefficients are on a similar scale.
{\smallskip}
                 \HLI{4} Coefficients \HLI{4}
             {\VBAR}      (b)          (B)            (b-B)     sqrt(diag(V_b-V_B))
             {\VBAR}       fe           re         Difference          S.E.
\HLI{13}{\PLUS}\HLI{64}
         ROE {\VBAR}    .1633808     .1151406        .0482401        .0077012
     CorpAge {\VBAR}    .0455443     .0161776        .0293667        .0110587
    CorpAge2 {\VBAR}   -.0002991    -.0002536       -.0000454               .
        Size {\VBAR}    .0750829     .0716512        .0034317        .0003929
LiquidityR{\tytilde}o {\VBAR}   -.0168089     -.017897        .0010881               .
       HHI10 {\VBAR}   -.1760706    -.1721294       -.0039412        .0076773
     PBRatio {\VBAR}    .0075511     .0069848        .0005663               .
        SGAe {\VBAR}    .0825913     .0545654        .0280259        .0077674
   CEOGender {\VBAR}    .0724868     .0656559        .0068309               .
      CEOAge {\VBAR}   -.0020626    -.0020026         -.00006               .
   CEOSalary {\VBAR}   -1.23e-07    -1.16e-07       -6.65e-09               .
    CEOShare {\VBAR}   -4.06e-09    -4.88e-09        8.23e-10        1.90e-10
 2004bn.year {\VBAR}   -.0196582     .0082732       -.0279314         .010887
   2005.year {\VBAR}   -.0321722     .0228997       -.0550719        .0224422
   2006.year {\VBAR}   -.0726744     .0114534       -.0841278        .0333425
   2007.year {\VBAR}   -.1294581    -.0149229       -.1145351        .0443114
   2008.year {\VBAR}   -.1597309    -.0190632       -.1406676         .055951
   2009.year {\VBAR}   -.1989022    -.0296251       -.1692771        .0670902
   2010.year {\VBAR}    -.233997    -.0350244       -.1989726        .0780411
   2011.year {\VBAR}   -.2807397    -.0552167        -.225523        .0894404
   2012.year {\VBAR}   -.3141874    -.0633026       -.2508848        .1009273
   2013.year {\VBAR}   -.3523648    -.0736676       -.2786972        .1122267
\HLI{13}{\BOTT}\HLI{64}
                         b = consistent under Ho and Ha; obtained from xtivreg
          B = inconsistent under Ha, efficient under Ho; obtained from xtivreg
{\smallskip}
    Test:  Ho:  difference in coefficients not systematic
{\smallskip}
                 chi2(19) = (b-B)'[(V_b-V_B){\caret}(-1)](b-B)
                          =   -48.67    chi2<0 ==> model fitted on these
                                        data fails to meet the asymptotic
                                        assumptions of the Hausman test;
                                        see suest for a generalized test
\nullskip
	\end{stlog}
\end{center}
As is shown in the result above, the statistics given by the {\bf Hausman Test} corresponds to a p-value of less than 0.01 percent. Thus, we can refuse the $H_0$, which is to say, we can rule out the potential existence of {\it Random Effect} upon 0.01 percent level and there is no need to do the further {\bf Breusch-Pagan Test}. 
Therefore, We stop here and would like to report the result of {\bf Fixed Effect Model}:
\begin{table}[htbp]\centering
\def\sym#1{\ifmmode^{#1}\else\(^{#1}\)\fi}
\caption{Result of Fixed Effect Model \label{tab:regressionFE}}
\begin{tabular}{l*{3}{c}}
\hline\hline
            &\multicolumn{1}{c}{(1)}&\multicolumn{1}{c}{(2)}&\multicolumn{1}{c}{(3)}\\
            &\multicolumn{1}{c}{FE}&\multicolumn{1}{c}{FE}&\multicolumn{1}{c}{IV-FE}\\
\hline
CorpAge     &      0.0321***&      0.0492***&      0.0455***\\
            &    (0.0065)   &    (0.0122)   &    (0.0135)   \\
[1em]
CorpAge2    &     -0.0004***&     -0.0002***&     -0.0003***\\
            &    (0.0000)   &    (0.0000)   &    (0.0001)   \\
[1em]
Size        &      0.0672***&      0.0741***&      0.0751***\\
            &    (0.0016)   &    (0.0025)   &    (0.0028)   \\
[1em]
ROE         &     -0.1411***&     -0.1171***&      0.1634** \\
            &    (0.0049)   &    (0.0067)   &    (0.0678)   \\
[1em]
LiquidityRatio&               &     -0.0169***&     -0.0168***\\
            &               &    (0.0005)   &    (0.0006)   \\
[1em]
HHI10       &               &     -0.0666***&     -0.1761***\\
            &               &    (0.0190)   &    (0.0336)   \\
[1em]
PBRatio     &               &      0.0059***&      0.0076***\\
            &               &    (0.0021)   &    (0.0023)   \\
[1em]
SGAe        &               &      0.0308***&      0.0826***\\
            &               &    (0.0051)   &    (0.0136)   \\
[1em]
CEOGender   &               &      0.0702***&      0.0725***\\
            &               &    (0.0188)   &    (0.0207)   \\
[1em]
CEOAge      &               &     -0.0018***&     -0.0021***\\
            &               &    (0.0007)   &    (0.0007)   \\
[1em]
CEOSalary   &               &     -0.0000***&     -0.0000***\\
            &               &    (0.0000)   &    (0.0000)   \\
[1em]
CEOShare    &               &     -0.0000***&     -0.0000***\\
            &               &    (0.0000)   &    (0.0000)   \\
\hline
\(N\)       &       21496   &        9946   &        9936   \\
adj. \(R^{2}\)&       0.088   &       0.160   &               \\
\hline\hline
\multicolumn{4}{l}{\footnotesize Standard errors in parentheses}\\
\multicolumn{4}{l}{\footnotesize * \(p<0.1\), ** \(p<0.05\), *** \(p<0.01\)}\\
\end{tabular}
\end{table}


\newpage
\section{Endogenous Problem}
It's arguable that all the main explaining variables we care have endogenous problem:
\begin{description}
	\item [CorpAge]: Older company my 
	\item [Size]:
	\item [ROE]:
\end{description}
\section{IV: }

\begin{appendix}
\section*{Appendix}
\subsection{Stata Do-file}
\subsection{Stata Log-file}
\end{appendix}
\end{document}

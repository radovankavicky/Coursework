\documentclass{article}
  \usepackage{amsmath}
  \usepackage{amssymb}
  \usepackage{graphicx}
  \usepackage{float}
  \usepackage{setspace}
  \usepackage{natbib}
  \usepackage{indentfirst}
  \usepackage{listings}
  \usepackage{textcomp}
  \usepackage{color}
  \usepackage{fontenc}
  \usepackage{accsupp}% http://ctan.org/pkg/accsupp
  \usepackage[colorlinks, linkcolor=NavyBlue, anchorcolor=Maroon, citecolor=TealBlue]{hyperref}%For hyper-link reference
  \usepackage{mdframed}
\setlength{\parindent}{2em}

\topmargin=-1.2cm \oddsidemargin=0.1cm \evensidemargin=0.1cm
\textwidth=16 true cm \textheight=23 true cm

\font\euler=EUSM10 \font\eulers=EUSM7

\begin{document}

\title{Financial Engineering\\Assignment $1^{\text{st}}$}
\author{{\normalsize SHENG Hao, 1401211818, via \LaTeX}}
\date{\today}

\maketitle

\def \Pr{{\rm Pr}}
\baselineskip 0.6cm


\section{Problem 1.}
The bond pays $4$ every half year. The yield of the bond is $12 \%$. Thus, we have the following equation:
\begin{equation}
  \frac{4}{(1 + \frac{12 \%}{2})} + \frac{4}{(1 + \frac{12 \%}{2})^2} + \frac{100+ 4}{(1 + \frac{12 \%}{2})^3} = \frac{4}{(1 + \frac{10 \%}{2})} + \frac{4}{(1 + \frac{10 \%}{2})^2} + 
\frac{100 + 4}{(1 + \frac{y}{2})^2}
\end{equation}

where $y$ is the 18-month zero rate. 

Solving this equation:
$$ y = 12.1 \% $$

Thus, the 18-month zero rate is $12.1 \%$.

\section{Problem 2.}
Consider the portfolio that long 2 of the 4\% bond and short 1 of the 8\% bond. Except for initial and ending year, all the cash flows cancels out. This makes this portfolio an equivalent to a 10-year zero coupon rate bond, i.g. thus they have the same yield. And since this portfolio sells at \$70 at year 0 and yield a return of \$100 at year 10, if we compound continuously, the 10-year zero rate would be:
\begin{equation}
   r=\frac{1}{10} \log \left(\frac{100}{70}\right) = 3.57 \% 
\end{equation}

\section{Problem 3.}
\subsection{(a)}
The selling price of the bond would be:
\begin{equation}
  B = \sum_{i = 1}^{5} \frac{8}{(1+ 11\%)^i} + \frac{100}{(1+11 \%)^5} = 88.9
\end{equation}
Thus, the duration:
\begin{equation}
  D = \sum_{i=1}^{5} i \left[ \frac{8}{(1 + 11 \%)^i B} \right] + 5 \left[ \frac{100}{(1+ 11 \%)^5 B} \right ] = 4.3 
\end{equation}

\subsection{(b)}
First we calculate the modified duration:
\begin{equation}
  D_{\text{m}} = 3.84
\end{equation}
Then, the effect led by yield change would be:
\begin{equation}
  \Delta B = - B D_{\text{mod}} \Delta y = - 88.9123 \cdot 3.84309 \cdot (- 0.2 \%) = 0.68
\end{equation}
That is a price increase about 0.68.

\section{Problem 4.}
\subsection{(a)}
Since no dividend is paid, 
\begin{equation}
  F_0 = e^{0.1} 40 = \$44.21
\end{equation}
And the initial value is always 0.

\subsection{(b)}
Denote $F(t,T)$ for forward price at time $t$ which will be delivered at time $T (t<T)$.
\begin{equation}
  F(0.5, 1) = e^{r(T-t)} = e^{0.1 \times 0.5} 45 = \$47.31
\end{equation}
And the value of the forward contract would be:
\begin{equation}
  (47.3072 - 44.2068) \times e^{-0.1 \times 0.5} = \$2.95
\end{equation}
\section{Problem 5.}
For a time-variant risk-free rate $r(t)$ and yield rate $q(t)$, both compounded continuously, the forward price for spot $t$ and delivered in $T$, $F(t, T)$ can be calculated as:
\begin{equation}
   F(t, T) = \exp \left( {\int_{t}^{T} (r(\tau) - q(\tau)) d\tau} \right) S(t) 
\end{equation}
Thus, the future price on Dec. $31^\text{st}$ would be:
\begin{equation}
  F = \exp \left( \frac{5}{12} \times 9 \% -\frac{2}{12} \times 5 \% - \frac{3}{12} \times 2 \% \right) \cdot 1300 = 1331.8
\end{equation}
\section{Problem 6.}
Denote $Q_T$ as $Q_T = \frac{1}{S_T}$. The payoff of the instrument is:
\begin{itemize}
  \item $Q_T\leq \frac{1}{180}, 1000$
  \item $\frac{1}{180} < Q_T < \frac{1}{90}, 1000(2-180Q_T)$
  \item $Q_T\geq \frac{1}{180}, 0$
\end{itemize}
It's the combination of the following three:
\begin{itemize}
  \item A short position in a put option of \$ 1000 with S=180 Yen/USD
  \begin{itemize}
    \item $Q_T\leq\frac{1}{180},0$
    \item $Q_T>\frac{1}{180}, 1000(1-180Q_T)$
  \end{itemize}
  \item A long position in a put option of \$ 2000 with S=90 Yen/USD
  \begin{itemize}
    \item $Q_T\leq\frac{1}{90},0$
    \item $Q_T>\frac{1}{180}, 1000(-2+180Q_T)$
  \end{itemize}
  \item A long position in a zero coupon rate bond with a face value of \$1000
\end{itemize}

\end{document}
